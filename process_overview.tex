\tikzstyle{block} = [rectangle, draw, fill=blue!20, text width=3cm, text centered, rounded corners, minimum height=1.5cm]
\tikzstyle{line} = [draw, very thick, color=black!50, -latex']

The general flow chart of the overall process flow can be seen in \autoref{full_flow}.
These process steps will be discussed within the following sections.
\begin{figure}[H]
	\centering
	\begin{tikzpicture}[node distance=2cm, thick,scale=0.6, every node/.style={transform shape}]
		%% Place nodes
		%active CMOS				
		\node [block] (isolation) at (4,20) {Isolation (STI)\\ \autoref{sti}};
		\node [block, below of=isolation] (well) {N-Well\\ \autoref{well}};
		\node [block, below of=well] (np) {n+ Implant\\ \autoref{nimplant}};
		\node [block, below of=np] (pp) {p+ Implant\\ \autoref{pimplant}};
		\node [block, below of=pp] (gate) {Gate contacts\\ \autoref{gate}};
		%post proces
		\node [block] (via1) at (8,12) {First vias\\ \autoref{via1}};
		\node [block, above of=via1] (metal1) {First metal\\ \autoref{metal1}};
		\node [block, above of=metal1] (via2) {Additional vias\\ \autoref{via2}};
		\node [block, above of=via2] (metal2) {Additional metal\\ \autoref{metal2}};
		\node (repeat) at (10.5,18.5) {Repeat};

		%% Draw edges
		\path [line] (isolation) -- (well);
		\path [line] (well) -- (np);
		\path [line] (np) -- (pp);
		\path [line] (pp) -- (gate);
		\path [line] (gate) -- (via1);
		\path [line] (via1) -- (metal1);
		\path [line] (metal1) -- (via2);
		\path [line] (via2) -- (metal2);
		\path [line] (metal2) -- +(3,0) -- +(3,-2) -- (via2);

		\draw[dotted] (2,10) rectangle (6,21);
		\node at (4,10.5) {CMOS process};
		\draw[dotted] (6,10) rectangle (12,21);
		\node at (8,10.5) {Interconnect};

		%\draw[dotted] (1.5,9) rectangle (10.5,21.5);
		%\node at (4,9.5) {Front-end processing};

		%\draw[dotted] (11,9) rectangle (15,21.5);
		%\node at (13,9.5) {Back-end processing};
	\end{tikzpicture}
	\caption{Frontend and backend process flow}
	\label{full_flow}
\end{figure}
The five overall process steps are part of an active part of the technology, while the final metal (respectively contact) layers will be used for making a contact between the logic gates and macro cells and making them available to the exterior world.

For this process p-substrate is the required basic substrate, but forks and modifications will be very well possible based on a Graphene substrate or alike, still under the LSPL.
The starting material is a p-type, <100> oriented silicon with a doping concentration of $\approx 9\times10^{14}cm^{-3}$.\\

\textbf{Reasons for using p-substrate}:\begin{itemize}
\item We can't use two different substrates for our design because in the design both PMOS and NMOS is present.
We have to choose which is more beneficial from fabrication point of view.
In general or say it's true that NMOS devices are always more in the Semiconductor Industry in comparison to PMOS devices.
For your reference-SRAM requires 6 transistors (4 NMOS, 2 PMOS).
\item Another reason for more number of NMOS is because of difference of mobility of electron and holes.
Electron mobility is almost twice of holes mobility and because of this ON-RESISTANCE of n-channel device is half of p-channel device with the same geometry and under the same operating conditions.
That means to achieve same impedance size of n-channel transistors is almost half of p-channel devices.
Same thing I can say in the different way that for same size of wafer, we can have more number of NMOS (means can perform more logical operation) in comparison to PMOS.
\end{itemize}

For the isolation (\autoref{sti})  in this design the STI approach is being chosen.
Shallow trench isolation (STI), also known as box isolation technique, is an integrated circuit feature which prevents electric current leakage between adjacent semiconductor device components.\footnote{https://www.google.com/patents/US7985656}
STI is generally used on CMOS process technology nodes of 250 nanometers and smaller.

\textbf{Reasons for using box isolation}:\begin{itemize}
\item We want to be forward compatible to future LibreSilicon nodes with a size of 100nm or smaller
\item It simplifies the construction of the gate and allows us to use Aluminum instead of Polysilicon for the gate contact
\end{itemize}

The interconnects and the gate electrode are being made using Aluminum.

\textbf{Reasons for using Aluminum}:\begin{itemize}
\item Aluminum is easy to etch compared to copper
\item It ain't doing "strange things" when making contact with doped areas, like copper does
\end{itemize}