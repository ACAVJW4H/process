\subsection{Etching silicon dioxide}
A very "selective" chemical for $SiO_2$ - i.e. does not etch silicon at all - is hydrofluoric acid (HF). If used directly such etchant has a too fast and aggresive action on the oxide, making very difficult the undercut and the linewidth control. For such reason, HF is universally used as a "buffered" solution, which can keep the etch rate low and constant, by moderating the PH level of the bath. This allows the etching time to be reliably correlated to the etching depth.

The industry standard buffered hydrofluoric acid solution (BHF\label{BHF}) has the following formulation:
\begin{itemize}
	\item 6 volumes of ammonium floride ($NH_4F$, 40\% solution)
	\item 1 volume of HF.
\end{itemize}
This can be prepared, for example, by mixing 113 g of $NH_4F$ in 170 ml of $H_2O$, and adding 28 ml of HF.\\
The etch rate at room temperature can range from 1000 to 2500 \r{A}/min (100-250nm/min).
This depends on the actual density of the oxide which, as an amorphous layer, can have a more compact structure (if thermally grown in is oxygen) or less compact (if grown by CVD).
The following etching reaction holds:
\begin{equation}
	SiO_2 + 6HF \rightarrow H_2SiF_6 + H_2O
\end{equation}\\
where $H_2SiF_6$ is water soluble.\\
A common buffered oxide etch solution comprises a 6:1 volume ratio of 40\% $NH_4F$ in water to 49\% HF in water. This solution will etch thermally grown oxide at approximately 2 nanometres per second at 25 degrees Celsius.\label{BHF_six_to_one}
\footnote{Wolf, S.; R.N. Tauber (1986). Silicon Processing for the VLSI Era: Volume 1 - Process Technology. pp. 532–533. ISBN 978-0-9616721-3-3} \\

Another popular etching formulation is the P-etch:

60 volumes of $H_2O$ + 3 vol. of HF + 2 vol. of $HNO_3$, that is: 300 ml of $H_2O$ + 15 ml of HF + 10 ml of $HNO_3$.

The P-etch action is strongly dependent on oxide density, as it results from the growth technique.
An example is reported in the literature\footnote{A. Pliskin, J.Vac.Sci Technol., vol. 14, p.1064, 1977}, indicating 120 \r{A}/min for thermal oxide and 250-700 \r{A}/min for sputtered oxide.

A slow etching bath is preferred for opening mask windows for a silicon substrate.
However, the etching process could be used just for removing the oxide film from the whole surface.
In this case the etching speed is not critical, and a fast solution can be used, such as HF diluited 1:10 in water.
The etching time can be easily evaluated by visually inspecting the surface.
Once the oxide film is removed, the metal-grey color of the silicon surface appears.

Sometimes a very light etch is required, for removing just a few atomic layers.
This is the case of surface cleaning and decontamination. HF diluited 1 : 50 in water can be used.
The etching speed will be around 70 \r{A} / min. For example, a typical 50 \r{A} "native" oxide on silicon can be removed with a 45 - 50 sec light etch.

\newpage

\subsection{Etching silicon nitride}
Thin films made of amorphous silicon nitride ($Si_3N_4$) are usually deposited by chemical vapour deposition from silane ($SiH_4$) and ammonia ($NH_3$).
Since they act as a barrier for water and sodium, they have a major role as passivation layers in microchip fabrication.
Patterned nitride layers are also used as a mask for spatially selective silicon oxide growth, and as an etch mask when $SiO_2$ masks cannot be used.

One example of the latter situation is given by the anisotropic etching of silicon in KOH.
The etching rate of $SiO_2$ in KOH is nearly 1000 times slower than the etching rate of silicon, and in most cases a $SiO_2$ mask can be used successfully.
However, a very deep selective etch may require a long etching time, and the 1000:1 etching rate ratio may result still too small to prevent the $SiO_2$ mask from being etched off before the process is completed.
In this circumstance $Si_3N_4$, thanks to its reduced etched rate, can successfully replace the oxide mask layer.

The wet etching of nitride films is often performed in concentrated hot orthophosphoric acid ($H_3PO_4$).
The bath temperature can range from 150\degree C to 180\degree C (boiling point) with a corresponding etch rate between 10 and 100 \r{A}/min. 
It is good practice to bring the vapours into contact with a cold surface and to drive the condensed liquid back into the etching bath.
This technique is referred to as "reflux".\label{chemistry_reflux}

The etching rates of silicon nitride, silicon oxide, and silicon in $H_3PO_4$ are respectively in the 50 : 5 : 1 ratio.
