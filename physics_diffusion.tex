\subsection{Diffusion (Doping)}

Although the diffusion process of donors and acceptors into the silicon crystal is a three dimensional process for simplicity we first only discuss the one dimensional mathematics for it in order to get a "simple" equation for the depth-time-temperature relation.

The diffusion coefficient is as well material as well as temperature dependent  and can be calculated with the following equation:
\begin{equation}
D = D_0 \cdot \exp\left(-\frac{E_a}{k \cdot T}\right)
\end{equation}
With $k=8.62 \cdot 10^{-5} \frac{eV}{K}$ being the Boltzman constant and in \autoref{absolute_diffusion_coefficients} we can see the $D_0$ and $E_a$ values for the most common materials\footnote{ISBN 3-8023-1588:Hoppe Bernhard, Mikroelektronik 2, Page 24, Table 2.1} which we can use within the further calculations for our well dimensioning phases. The temperature usually is in the area of $1000\degree C$ or in Kelvin $1273.15\degree K$.
\begin{table}[H]
	\centering
	\begin{tabular}{|c|c|c|}
		\hline
		Element &
		$D_0$ $\left[\frac{cm^2}{s}\right]$ &
		$E_a$ $\left[eV\right]$ \\
		\hline
		P &
		10.50 &
		3.69 \\
		\hline
		As &
		0.32 &
		3.56 \\
		\hline
		Sb &
		5.60 &
		3.95 \\
		\hline
		B &
		10.50 &
		3.69 \\
		\hline
		Al &
		8.00 &
		3.47 \\
		\hline
		Ga &
		3.60 &
		3.51 \\
		\hline
		Cu &
		0.0025 &
		0.65 \\
		\hline
	\end{tabular}
	\label{absolute_diffusion_coefficients}
	\caption{$D_0$ and $E_a$ values for Boron and Phosphorus}
\end{table}