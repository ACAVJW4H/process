\subsection{Isolation}
For the isolation (\autoref{sti})  in this design the STI approach is being chosen.
Shallow trench isolation (STI), also known as box isolation technique, is an integrated circuit feature which prevents electric current leakage between adjacent semiconductor device components.\footnote{\url{https://www.google.com/patents/US7985656}}
STI is generally used on CMOS process technology nodes of 250 nanometers and smaller.\\

\textbf{Reasons for using box isolation}:\begin{itemize}
\item We want to be forward compatible to future LibreSilicon nodes with a size of 100nm or smaller
\item It simplifies the construction of the gate and allows us to use Aluminum instead of Polysilicon for the gate contact
\end{itemize}

Issues we have to keep in mind is that the depth is not uniform and can variate strongly within a $2 \mu m$ range!
This means we have to make the well at least "deep enough" at the shallowest place, so that it provides adequate isolation between the transistors everywhere on the die.

One way to reduce the variation in depth is to have a uniform width of the isolation.

Also the non-uniform thickness of the oxide is a problem. \\

CMP (Chemical Mechanical Planarization) for evening the oxide out is being chosen, because the hard mask can be removed in the very same process step.