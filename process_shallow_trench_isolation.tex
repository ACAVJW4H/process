\subsection{Shallow trench isolation}\label{sti}
Shallow trench isolation (STI), also known as box isolation technique, is an integrated circuit feature which prevents electric current leakage between adjacent semiconductor device components.
STI is generally used on CMOS process technology nodes of 250 nanometers and smaller.
Older CMOS technologies and non-MOS technologies commonly use isolation based on LOCOS.
\footnote{Quirk, Michael \& Julian Serda (2001). Semiconductor Manufacturing Technology: Instructor's Manual Archived September 28, 2007, at the Wayback Machine., p. 25.} \\
The geometry of a substrate with STI implemented can be seen in \autoref{sti_target}.

\begin{figure}[H]
	\centering
	\begin{tikzpicture}[node distance = 3cm, auto, thick,scale=\CrossAndTopSectionBig, every node/.style={transform shape}]
		% substrate
\fill[substrate] (0,0) rectangle (20,2);
\node at (2,0.5) {Silicon substrate};
%trenches
\fill[isolationoxide] (0,0.75) rectangle (1,2);
\fill[isolationoxide] (8.5,0.75) rectangle (11.5,2);
\fill[isolationoxide] (19,0.75) rectangle (20,2);
	\end{tikzpicture}
	\begin{tikzpicture}[node distance = 3cm, auto, thick,scale=\CrossAndTopSectionBig, every node/.style={transform shape}]
		% substrate
\fill[YellowOrange] (0,0) rectangle (20,12);
% trench area
\fill[DarkGray] (0,0) rectangle (1,12);
\fill[DarkGray] (8.5,0) rectangle (11.5,12);
\fill[DarkGray] (19,0) rectangle (20,12);
\fill[DarkGray] (0,0) rectangle (20,1.25);
\fill[DarkGray] (0,7.5) rectangle (20,12);
	\end{tikzpicture}
	\caption{Shallow trench isolation target geometry}
	\label{sti_target}
\end{figure}

We choose the STI approach because we wanna scale the technology node down in the future below 250nm and wanna ensure backwards compatibility of our process.

As can be seen in \autoref{cmos_nutshell}, the n-well and the STI trench are supposed to have approximately the same depth.
Because the n-well will be $\approx 4 \mu m$ in depth (\autoref{well}) we have to match this with our trench depth.

\subsubsection{Pad oxide}
\subsubsection{Etching}
Dry etching (RIE)