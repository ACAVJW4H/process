\newcommand{\addAlignmentCross}[2]{
	\draw[line width=1.5mm, active,opacity=\OpacityLayout] (#1,#2+0.5) -- (#1+1,#2+0.5);
	\draw[line width=1.5mm, active,opacity=\OpacityLayout] (#1+0.5,#2) -- (#1+0.5,#2+1);
}

\subsection{Alignment strategy}
When having multiple layers exposed after each other, there is the problem on how to make sure that for instance vias are actually making contact with the below wire and the wire above.
For this purpose we have to align the masks, in order to avoid issues like shown in \autoref{what_can_go_wrong_with_alignment}

\begin{figure}[H]
	\centering
	\begin{tikzpicture}
		\fill[metal1,opacity=\OpacityLayout] (1,1) rectangle (2,4);
		\fill[via1,opacity=\OpacityLayout] (2.25,2.25) rectangle (2.75,2.75);
		\fill[metal2,opacity=\OpacityLayout] (1,1) rectangle (4,2);
	\end{tikzpicture} 
	\begin{tikzpicture}
		\fill[metal1,opacity=\OpacityLayout] (1,1) rectangle (2,4);
		\fill[via1,opacity=\OpacityLayout] (1.25,1.25) rectangle (1.75,1.75);
		\fill[metal2,opacity=\OpacityLayout] (2,1) rectangle (5,2);
	\end{tikzpicture}
	\caption{What can go wrong with alignment}
	\label{what_can_go_wrong_with_alignment}
\end{figure}

In \autoref{what_can_go_wrong_with_alignment} we can see how the wires and vias are missing each other because of an exposure offset and the via is going nowhere in the best case and creates an (of course) undesired short circuit in the worst case. This has to be avoided by using alignment.

We have decided to use backside alignment because shining through the wafer from behind with infrared and finding an orientation marker isn't a problem with our simple CMOS process.\footnote{\url{https://patents.google.com/patent/US6376329}}

The stepper machine at HKUST\footnote{http://www.nff.ust.hk/en/equipment-and-process/equipment-list/photolithography-module.html} does that for us.

We need to add orientation cross hairs onto the layout edge in order to identify them during alignment

\begin{figure}[H]
	\centering
	\begin{tikzpicture}
		\fill[metal1,opacity=\OpacityLayout] (1,1) rectangle (2,4);
		\fill[via1,opacity=\OpacityLayout] (1.25,1.25) rectangle (1.75,1.75);
		\fill[metal2,opacity=\OpacityLayout] (1,1) rectangle (4,2);
		\draw[dotted] (0.0,0) rectangle (5,5);
		% cross one
		\addAlignmentCross{0}{0}
		% cross two
		\addAlignmentCross{4}{4}
		% cross three
		\addAlignmentCross{4}{0}
		% cross four
		\addAlignmentCross{0}{4}
	\end{tikzpicture}
	\caption{Mask alignment with markers}
\end{figure}

Using the STI (active) mask for the cross hair structure is a good choice, because it's the lowest layer, will never be covered by any additional material and gives us a good contrast because it's silicon next to silicon dioxide.

\begin{figure}[H]
	\centering
	\begin{tikzpicture}
		\fill[metal1,opacity=\OpacityLayout] (1,1) rectangle (2,4);
		\draw[dotted] (0.0,0) rectangle (5,5);
		% cross one
		\addAlignmentCross{0}{0}
		% cross two
		\addAlignmentCross{4}{4}
		% cross three
		\addAlignmentCross{4}{0}
		% cross four
		\addAlignmentCross{0}{4}
	\end{tikzpicture} 
	\begin{tikzpicture}
		\fill[via1,opacity=\OpacityLayout] (1.25,1.25) rectangle (1.75,1.75);
		\draw[dotted] (0.0,0) rectangle (5,5);
		% cross one
		\addAlignmentCross{0}{0}
		% cross two
		\addAlignmentCross{4}{4}
		% cross three
		\addAlignmentCross{4}{0}
		% cross four
		\addAlignmentCross{0}{4}
	\end{tikzpicture} 
	\begin{tikzpicture}
		\fill[metal2,opacity=\OpacityLayout] (1,1) rectangle (4,2);
		\draw[dotted] (0.0,0) rectangle (5,5);
		% cross one
		\addAlignmentCross{0}{0}
		% cross two
		\addAlignmentCross{4}{4}
		% cross three
		\addAlignmentCross{4}{0}
		% cross four
		\addAlignmentCross{0}{4}
	\end{tikzpicture}
	\caption{Layout masks with alignment markers}
\end{figure}

