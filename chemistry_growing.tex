\subsection{Growing silicon nitride}\label{chemistry_growing_nitride}
In order to grow a high quality layer of silicon nitride on top of a silicon wafer which is adapted to be patterned and to serve as a mask for diffusion or implantation of selected impurities, the wafer is best put into a chamber evacuated to a pressure less than about 1 Torr and heated between 650 and 900 \degree C.
A gaseous mixture comprising primarily of ammonia and a silicon compound, having a ratio of relative concentrations in the range on 4:1 and 20:1 \footnote{\url{http://www.freepatentsonline.com/4395438.html}}, is flooded into that chamber with a silicon compound flow rate of greater than approximately 12 cubic centimeters per minute.
The growth rate will be around 50 Angstroms per minute.
That setup is called Low-Pressure Chemical Vapor Deposition (LPCVD), which is commonly available in basically any semiconductor manufacturing plant or laboratory.
