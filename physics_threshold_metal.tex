\subsubsection{Threshold voltage with metal gate ($V_T$)}
\begin{equation}
V_{FB}
=
\phi_{MS} - \frac{Q_{SS}}{C_{ox}}
=
\phi_{M} - \phi_{S} - \frac{Q_{SS}}{C_{ox}}
\end{equation}
and
\begin{equation}
\phi_{S}
=
\chi + \frac{E_g}{2 q} + \phi_F 
\end{equation}
we get
\begin{equation}
V_{FB}
=
\phi_{M} -  \left(\chi + \frac{E_g}{2 q} + \phi_F \right) - \frac{Q_{SS}}{C_{ox}}
\end{equation}

And because of the simplifications we did to $F_{FB}$ which essentially led to $F_{FB}=\phi_{MS}$ we get to:
\begin{equation}
V_T = V_{t-mos} + \phi_{MS}
\end{equation}
\begin{equation}
V_T = 2 \phi_F + \frac{Q_b}{C_{ox}} + \phi_{MS}
\end{equation}
\begin{equation}
V_T = 2 \phi_F + \frac{2 \sqrt{\epsilon_{Si}\cdot q\cdot \left| \phi_F \right| \cdot N_{implant} }}{C_{ox}} + \phi_{MS}
\end{equation}
\begin{equation}
V_T = 2 \phi_F + \frac{2 \sqrt{\epsilon_{Si}\cdot q\cdot \left| \phi_F \right| \cdot N_{implant} }}{C_{ox}} + \phi_{M} -  \left(\chi + \frac{E_g}{2 q} + \phi_F \right) - \frac{Q_{SS}}{C_{ox}}
\end{equation}
\begin{equation}
V_T = 2 \phi_F + \frac{2 \sqrt{\epsilon_{Si}\cdot q\cdot \left| \phi_F \right| \cdot N_{implant} }}{C_{ox}} + \phi_{M} -  \chi - \frac{E_g}{2 q} - \phi_F - \frac{Q_{SS}}{C_{ox}}
\end{equation}
\begin{equation}
V_T = \phi_F + \frac{2 \sqrt{\epsilon_{Si}\cdot q\cdot \left| \phi_F \right| \cdot N_{implant} }}{C_{ox}} + \phi_{M} -  \chi - \frac{E_g}{2 q} - \frac{Q_{SS}}{C_{ox}}
\end{equation}

The contact potential from the Aluminum contact to the surface of the gate (silicon below the oxide) is fixed for $T=300\degree K$:
\begin{equation}
\phi_{M} -  \chi - \frac{E_g}{2 q} = 4.1V - 4.05V - \frac{1.12eV}{2 q} = 4.1V - 4.05V - 0.56V = -0.51V
\end{equation}

From that we get
\begin{equation}
V_T = \phi_F + \frac{2 \sqrt{\epsilon_{Si}\cdot q \cdot \left| \phi_F \right| \cdot N_{implant} }}{C_{ox}} - 0.51V
\end{equation}

Now we can calculate the thresholds for P substrate ($V_{Tp}$) and N substrate  ($V_{Tn}$), respectively the wells we build on unpredoped substrated, which makes the equation for single-doped substrate valid for both wells with
\begin{equation}
\phi_{Fn}
=
V_{th} \cdot ln\left(\frac{N_i}{N_{implant}}\right)
\end{equation}
\begin{equation}
\phi_{Fp}
=
V_{th} \cdot ln\left(\frac{N_{implant}}{N_i}\right)
\end{equation}

Which brings us to the equations for the N-channel and P-channel thresholds:

(N-Channel MOSFETs are built on p-substrate)
\begin{equation}
V_{Tn} = \phi_{Fp} + \frac{2 \sqrt{\epsilon_{Si}\cdot q \cdot \left| \phi_{Fp} \right| \cdot N_{implant} }}{C_{ox}} - 0.51V
\end{equation}

(P-Channel MOSFETs are built on n-substrate)
\begin{equation}
V_{Tp} = \phi_{Fn} + \frac{2 \sqrt{\epsilon_{Si}\cdot q \cdot \left| \phi_{Fn} \right| \cdot N_{implant} }}{C_{ox}} - 0.51V
\end{equation}

This equation will be used further on to find the optimum gate oxide thickness for our transistors.


