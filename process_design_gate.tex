\subsection{MOS gate}
The wires and the MOS gates can be modeled as an RC low-pass filter, for that reason the time constant $\tau=RC$ is limiting the clock frequency of our CMOS circuitry.
However, we can't just make the gate ultra thin, because lowered capacity also creates lowered thresholds, which will be a very big problem, as soon as we go into the 100nm range and lower.

In \autoref{nmos_gate_dimensioning} we defined the capacity of the gate to be $C_{ox} \leq 305.7 \frac{nF}{cm^2}$ which leads to the oxide capacity-thickness constraint with the variables being:
\begin{itemize}
\item $\epsilon_0 = 8.85 \cdot 10^{-14}\frac{F}{cm}. $ is the electric permittivity in vacuum
\item $\epsilon_{ox} =3.9 \cdot \epsilon_0$ is the relative permittivity of silicon dioxide
\end{itemize}
and the constraint being for the oxide thickness which is only silicon dioxide for now
\begin{equation}
C_{ox}
=
\frac{\epsilon_{ox}}{t_{ox}}
\end{equation}
\begin{equation}
t_{ox}
=
\frac{\epsilon_{ox}}{C_{ox}}
\end{equation}

And with numbers we get
\begin{equation}
t_{ox}
=
\frac{3.9 \cdot 8.85 \cdot 10^{-14}\frac{F}{cm}}{305.7 \frac{nF}{cm^2}}
=
1.129 \cdot 10^{-6} cm
=
11.29nm
\end{equation}

That's still doable with a precision high enough when using dry oxidation and a temperature of 1000\degree Celsius.