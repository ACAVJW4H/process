\subsection{Threshold voltage ($V_T$)}\label{transistor_threshold_calculation}
The formula for calculating the threshold voltage of a MOS device is the following:
\begin{equation}
V_T = V_{t-mos} + V_{FB}
\end{equation}
where $V_{t-mos}$ is the ideal threshold voltage of an ideal MOS capacitor and $V_{FB}$ is what is termed flat-band voltage and $V_{t-mos}$ is the threshold.
The MOS threshold voltage, $V_{t-mos}$ is calculated by considering the MOS capacitor structure that form the gate of the MOS transistor.

The ideal threshold voltage may be expressed as:
\begin{equation}
V_{t-mos}=2 \phi_F + \frac{Q_b}{C_{ox}}
\end{equation}
\begin{equation}
Q_b
=
\sqrt{2 \epsilon_{Si} q N_A ( 2 \phi_F + V_{SB}) }
\end{equation}
where $C_{ox}$ is the oxide capacitance and $Q_b$ which is called the bulk charge term.\\

The bulk potential is given by:
\begin{equation}
\phi_F
=
V_{th} ln\left(\frac{p}{N_i}\right)
=
-V_{th} ln\left(\frac{n}{N_i}\right)
\end{equation}
As can be seen from the above equations, the bulk potential is positive for p-type substrates and negative for n-type substrates.
$V_{th}$ is the thermal voltage.\footnote{\url{https://en.wikipedia.org/wiki/Boltzmann_constant\#Role_in_semiconductor_physics:_the_thermal_voltage}}
\begin{equation}
V_{th} = \frac{k T}{q} 
\end{equation}

Since we connect bulk and source and are using p-substrate we can simplify the equation to become
\begin{equation}
Q_b
=
2 \sqrt{\epsilon_{Si} q N_A V_{th} ln\left(\frac{N_A}{N_i}\right) }
\end{equation}


$V_{FB}$, is given by:
\begin{equation}
V_{FB}
=
\phi_{MS}-\frac{Q_f}{C_{ox}}-\frac{1}{C_{ox}}\int_{0}^{t_{ox}}\frac{x}{x_{ox}}\rho(x) dx
\end{equation}

We assume the oxide doesn't contain any charge, which makes the equation fall together into
\begin{equation}
\rho(x)=0
\Rightarrow
V_{FB}
=
\phi_{MS}-\frac{Q_f}{C_{ox}}
\end{equation}
with
\begin{equation}
\phi_{MS}
=
\phi_{M} - \phi_{S}
=
\phi_{M} -  \left(\chi + \frac{E_g}{2} + \phi_F \right)
\end{equation}

We now can put the equation together:
\begin{equation}
V_T = V_{t-mos} + V_{FB}
\end{equation}
\begin{equation}
V_{t-mos}=2 \phi_F + \frac{Q_b}{C_{ox}}
\Rightarrow
V_T = 2 \phi_F + \frac{Q_b}{C_{ox}} + V_{FB}
\end{equation}

With the variables being the following:
\begin{itemize}
\item $N_A \approx 9\times10^{14}cm^{-3}$ is the substrate doping (of our chosen basis substrate. also see \autoref{process_overview})
\item $N_i$ is the carrier concentration in intrinsic (undoped) silicon. $N_i$ is equal to $1.45 \times 10^{10} cm^{-3}$ at 300\degree K
\item $\phi_M = 4.1 V$ is the "work function" of our metal at the gate (multiplied with q) (Only Aluminum is practical for this process)
\item $E_g(T) = E_g(0) - \frac{\alpha T^2}{T+\beta} = 1.166 - 4.73 \cdot 10^{-4} \cdot \frac{T^2}{T+636} [eV]$ is the band gap energy of silicon at a given temperature\footnote{https://ecee.colorado.edu/~bart/book/eband5.htm} for which the parameters can be taken from \autoref{band_gap_parameters}
\begin{table}[H]
\centering
\begin{tabular}{|c|c|c|c|}
\hline
{} &
\textbf{Germanium} &
\textbf{Silicon} &
\textbf{GaAs} \\
\hline
$Eg(0) [eV]$ &
0.7437 &
1.166 &
1.519 \\
\hline
$\alpha [eV/K]$ &
4.77 x 10-4 &
4.73 x 10-4 &
5.41 x 10-4 \\
\hline
$\beta [K]$ &
235 &
636 &
204 \\
\hline
\end{tabular}
\caption{Band cap energy parameters}
\label{band_gap_parameters}
\end{table}
\item $C_{ox} \left[\frac{F}{cm^2}\right]$ is the capacity of the gate oxide
\item $Q_{fc}$ is the surface state charge (which we assume zero for a simple CMOS circuit without NVM devices)
\item $\epsilon_0 = 8.85 \cdot 10−14 \frac{F}{cm}. $ is the electric permittivity in vacuum
\item $\epsilon_{Si} =11.68 \cdot \epsilon_0$ is the relative permittivity of silicon
\item $\epsilon_{ox} = 3.9 \cdot \epsilon_0$ is the relative permittivity of silicon oxide
\item $t_{ox} [cm]$ is the thickness of the oxide layer in cm
\item $\chi = 4.05 eV$ is the electron affinity of a silicon crystal surface\footnote{\url{https://en.wikipedia.org/wiki/Electron_affinity}}
\item $k=8.6173303 \cdot 10^{-5} \frac{eV}{K}$ is the Boltzmann constant
\item $q=1.602 \cdot 10^{-19} C$ is the elementary charge
\end{itemize}