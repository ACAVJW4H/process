\subsection{Threshold voltage ($V_T$)}\label{transistor_threshold_calculation}
The formula for calculating the threshold voltage of a MOS device is the following:
\begin{equation}
V_T = V_{t-mos} + V_{FB}
\end{equation}
where $V_{t-mos}$ is the ideal threshold voltage of an ideal MOS capacitor and $V_{FB}$ is what is termed flat-band voltage and $V_{t-mos}$ is the threshold.
The MOS threshold voltage, $V_{t-mos}$ is calculated by considering the MOS capacitor structure that form the gate of the MOS transistor.

The ideal threshold voltage may be expressed as:
\begin{equation}
V_{t-mos}=2 \phi_F + \frac{Q_b}{C_{ox}}
\end{equation}
\begin{equation}
Q_b
=
\sqrt{2 \epsilon_{Si} \cdot q \cdot N_{implant}  \cdot  ( \left| 2 \phi_F \right| + V_{SB}) }
\end{equation}
where $C_{ox}$ is the oxide capacitance and $Q_b$ which is called the bulk charge term.\\

The bulk potential is given by:
\begin{equation}
\phi_F
=
V_{th} \cdot ln\left(\frac{p}{N_i}\right)
=
V_{th} \cdot ln\left(\frac{N_i}{n}\right)
\end{equation}

$V_{th}$ is the thermal voltage.\footnote{\url{https://en.wikipedia.org/wiki/Boltzmann_constant\#Role_in_semiconductor_physics:_the_thermal_voltage}}
\begin{equation}
V_{th} = \frac{k T}{q} \approx 0.026 \frac{J}{C} = 0.026 V = 26mV
\end{equation}
With the variables being:
\begin{itemize}
\item $k=1.38064852\cdot 10{-23}  \frac{J}{K}$ is the Boltzmann constant
\item $q=1.602 \cdot 10^{-19} C$ is the elementary charge
\item $T= 300 \degree K$ the temperature, which we assume to be the room temperature for simplicity further on in this document as well.
\end{itemize}

\begin{mdframed}[linewidth=2pt,linecolor=red]
We can directly switch $\frac{J}{C}$ with Volts because these two units are equal!\footnote{\url{https://en.wikipedia.org/wiki/Volt}}
Also $V_{th}$ will be treated as a constant for any further calculations within this document.
\end{mdframed}

Since we connect bulk and source $V_{SB}=0$ we can simplify the equation to become
\begin{equation}
Q_b
=
\sqrt{2\cdot\epsilon_{Si}\cdot q\cdot N_{implant} \cdot  ( \left| 2 \cdot \phi_F \right|) }
\end{equation}
\begin{equation}
Q_b
=
2\cdot\sqrt{\epsilon_{Si}\cdot q\cdot N_{implant}\cdot \left| \phi_F \right| }
\end{equation}
\begin{equation}
Q_b
=
2 \sqrt{\epsilon_{Si}\cdot q\cdot V_{th}\cdot ln\left(\frac{N_{implant}}{N_{inert}}\right) \cdot N_{implant} }
\end{equation}


$V_{FB}$, is given by:
\begin{equation}
V_{FB}
=
\phi_{MS}-\frac{Q_f}{C_{ox}}-\frac{1}{C_{ox}}\int_{0}^{t_{ox}}\frac{x}{x_{ox}}\rho(x) dx
\end{equation}

Because we're not yet dealing with non-volatile memory devices which contain an oxide surface state charge we can just set $Q_f=0$ as well as $\rho(x)=0$
\begin{equation}
V_{FB}
=
\phi_{MS}
\end{equation}
with
\begin{equation}
V_{FB}
=
\phi_{MS}
=
\phi_{M} - \phi_{S}
=
\phi_{M} -  \left(\chi + \frac{E_g}{2} + \phi_F \right)
\end{equation}

And because of the simplifications we did to $F_{FB}$ which essentially led to $F_{FB}=\phi_{MS}$ we get to:
\begin{equation}
V_T = V_{t-mos} + \phi_{MS}
\end{equation}

That's our target equation, in which the capacity has the unit $\frac{F}{m^2}$ and $V_T$ has the unit Volt:
\begin{equation}
\boxed{
V_T= \frac{1.66 \cdot 10^{-7}}{C_{ox}}-0.094
}
\end{equation}

This equation will be used further on to find the optimum gate oxide thickness for our transistors.
