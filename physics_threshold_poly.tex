\subsubsection{Threshold voltage with poly silicon gate ($V_T$)}
The formula for calculating the threshold voltage of a MOS device is the following:
\begin{equation}
V_T = V_{t-mos} + V_{FB}
\end{equation}
where $V_{t-mos}$ is the threshold voltage of an ideal MOS capacitor, $V_{FB}$ is the flat-band voltage and $V_{t-mos}$ is the threshold.
The MOS threshold voltage, $V_{t-mos}$ is calculated by considering the MOS capacitor structure that form the gate of the MOS transistor.

The ideal threshold voltage may be expressed as:
\begin{equation}
V_{t-mos}=2 \phi_F + \frac{Q_b}{C_{ox}}
\end{equation}
\begin{equation}
Q_b
=
\sqrt{2 \epsilon_{Si} \cdot q \cdot N  \cdot  ( \left| 2 \phi_F \right| + V_{SB}) }
\end{equation}
where $C_{ox}$ is the oxide capacitance and $Q_b$ which is called the bulk charge term.\\

Since we connect bulk and source $V_{SB}=0$ we can simplify the equation to become
\begin{equation}
Q_b
=
\sqrt{2\cdot\epsilon_{Si}\cdot q\cdot N \cdot  ( \left| 2 \cdot \phi_F \right|) }
\end{equation}
\begin{equation}
Q_b
=
2\cdot\sqrt{\epsilon_{Si}\cdot q\cdot N \cdot \left| \phi_F \right| }
\end{equation}


$V_{FB}$ is the flat band voltage and is given by:
\begin{equation}
V_{FB}
=
\phi_{MS}-\frac{Q_{SS}}{C_{ox}}-\frac{1}{C_{ox}}\int_{0}^{t_{ox}}\frac{x}{x_{ox}}\rho(x) dx
\end{equation}

Because we're not yet dealing with non-volatile memory devices which contain an oxide surface state charge we can just $\rho(x)=0$.
$Q_{SS}$ is a value which has to be measured.

\begin{equation}
V_{FB}
=
\phi_{MS} - \frac{Q_{SS}}{C_{ox}}
\end{equation}
with
\begin{equation}
V_{FB}
=
\phi_{MS} - \frac{Q_{SS}}{C_{ox}}
=
\phi_{M} - \phi_{S} - \frac{Q_{SS}}{C_{ox}}
\end{equation}

The term $\phi_{MS}$ is the work function difference between the gate material and the silicon substrate ($\phi_{gate}-\phi_{Si}$), which may be calculated for an n+ gate over a p substrate

\begin{equation}
\phi_{MSp}
=
-(\frac{E_g}{2}+\phi_{Fp})
\end{equation}

and for an n+ poly gate on an n-substrate
\begin{equation}
\phi_{MSn}
=
-(\frac{E_g}{2}-\phi_{Fn})
\end{equation}

Now we can calculate the thresholds for P substrate ($V_{Tp}$) and N substrate  ($V_{Tn}$), respectively the wells we build on unpredoped substrated, which makes the equation for single-doped substrate valid for both wells.

(N-Channel MOSFETs are built on p-substrate)
\begin{equation}
V_{Tn} = 2 \cdot \phi_{Fp} + \frac{2 \sqrt{\epsilon_{Si}\cdot q \cdot \left| \phi_{Fp} \right| \cdot N_p}}{C_{ox}} + \phi_{MSp} - \frac{Q_{SS}}{C_{ox}}
\end{equation}
\begin{equation}
V_{Tn} = 2 \cdot \phi_{Fp} + \frac{2 \sqrt{\epsilon_{Si}\cdot q \cdot \left| \phi_{Fp} \right| \cdot N_p}}{C_{ox}} -\frac{E_g}{2} - \phi_{Fp} - \frac{Q_{SS}}{C_{ox}}
\end{equation}
\begin{equation}
V_{Tn} = \phi_{Fp} + \frac{2 \sqrt{\epsilon_{Si}\cdot q \cdot \left| \phi_{Fp} \right| \cdot N_p}}{C_{ox}} -\frac{E_g}{2} - \frac{Q_{SS}}{C_{ox}}
\end{equation}

(P-Channel MOSFETs are built on n-substrate)
\begin{equation}
V_{Tp} = 2 \cdot \phi_{Fn} + \frac{2 \sqrt{\epsilon_{Si}\cdot q \cdot \left| \phi_{Fn} \right| \cdot N_n}}{C_{ox}} + \phi_{MSn} - \frac{Q_{SS}}{C_{ox}}
\end{equation}
\begin{equation}
V_{Tp} = 2 \cdot \phi_{Fn} + \frac{2 \sqrt{\epsilon_{Si}\cdot q \cdot \left| \phi_{Fn} \right| \cdot N_n}}{C_{ox}} -\frac{E_g}{2} + \phi_{Fn} - \frac{Q_{SS}}{C_{ox}}
\end{equation}
\begin{equation}
V_{Tp} = 3 \cdot \phi_{Fn} + \frac{2 \sqrt{\epsilon_{Si}\cdot q \cdot \left| \phi_{Fn} \right| \cdot N_n}}{C_{ox}} -\frac{E_g}{2} - \frac{Q_{SS}}{C_{ox}}
\end{equation}

This equation will be used further on to find the optimum gate oxide thickness for our transistors.


