\subsection{NMOS threshold}\label{nmos_gate_dimensioning}
First we take a look at the worst case of 4 stacked NMOS transistors, which is the highest stacking amount which will be possible in technologies relying on this process.

%%  ************    LibreSilicon's StdCellLibrary   *******************
%%
%%  Organisation:   Chipforge
%%                  Germany / European Union
%%
%%  Profile:        Chipforge focus on fine System-on-Chip Cores in
%%                  Verilog HDL Code which are easy understandable and
%%                  adjustable. For further information see
%%                          www.chipforge.org
%%                  there are projects from small cores up to PCBs, too.
%%
%%  File:           StdCellLib/Documents/LaTeX/schematic_AND4.tex
%%
%%  Purpose:        Schematic File for AND4
%%
%%  ************    LaTeX with circdia.sty package      ***************
%%
%%  ///////////////////////////////////////////////////////////////////
%%
%%  Copyright (c) 2018 by chipforge <hsank@nospam.chipforge.org>
%%  All rights reserved.
%%
%%      This Standard Cell Library is licensed under the Libre Silicon
%%      public license; you can redistribute it and/or modify it under
%%      the terms of the Libre Silicon public license as published by
%%      the Libre Silicon alliance, either version 1 of the License, or
%%      (at your option) any later version.
%%
%%      This design is distributed in the hope that it will be useful,
%%      but WITHOUT ANY WARRANTY; without even the implied warranty of
%%      MERCHANTABILITY or FITNESS FOR A PARTICULAR PURPOSE.
%%      See the Libre Silicon Public License for more details.
%%
%%  ///////////////////////////////////////////////////////////////////
    \begin{figure}[H]
	\centering
            \begin{circuitdiagram}{50}{33}
            \pin{2}{2.5}{L}{A}  % pin A, n-channel below
            \pin{2}{8.5}{L}{B}  % pin B, n-channel middle
            \pin{2}{14.5}{L}{C} % pin C, n-channel middle
            \pin{2}{20.5}{L}{D} % pin D, n-channel above
            \pin{2}{29.5}{L}{A} % pin A, p-channel left
            \pin{14}{29.5}{L}{B} % pin B, p-channel middle
            \pin{26}{29.5}{L}{C} % pin C, p-channel middle
            \pin{38}{29.5}{L}{D} % pin D, p-channel right
            \trans{nenh*}{6}{4}{R}{}{}  % nmos below
            \trans{nenh*}{6}{10}{R}{}{} % nmos middle
            \trans{nenh*}{6}{16}{R}{}{} % nmos middle
            \trans{nenh*}{6}{22}{R}{}{} % nmos above
            \trans{nenh*}{54}{22}{R}{}{} % nmos inverter
            \trans{penh*}{6}{28}{R}{}{} % pmos left
            \trans{penh*}{18}{28}{R}{}{} % pmos middle
            \trans{penh*}{30}{28}{R}{}{} % pmos middle
            \trans{penh*}{42}{28}{R}{}{} % pmos right
            \trans{penh*}{54}{28}{R}{}{} % pmos inverter
            \wire{8}{6}{8}{8}     % wire between nmos
            \ground{8}{0.5}{D}  % ground below nmos
            \ground{56}{18.5}{D}  % ground below inverter
            \power{8}{31.5}{U}{}  % power above left pmos
            \power{20}{31.5}{U}{}  % power above middle pmos
            \power{32}{31.5}{U}{}  % power above middle pmos
            \power{44}{31.5}{U}{}  % power above right pmos
            \power{56}{31.5}{U}{}  % power above inverter
            \wire{8}{12}{8}{14}     % wire between nmos
            \wire{8}{18}{8}{20}     % wire between nmos and pmos
            \wire{20}{25}{20}{26}
            \wire{32}{25}{32}{26}
            \wire{44}{25}{44}{26}
            \wire{8}{25}{51}{25}    % wire before inverter gate
            \junct{8}{25}
            \junct{20}{25}
            \junct{32}{25}
            \junct{44}{25}
            \junct{56}{25}
            \wire{51}{29.5}{51}{20.5}   % wire at inverter gates
            \junct{51}{25}
            \wire{56}{25}{58}{25}    % wire before pin Z
            \pin{59}{25}{R}{Z}  % pin Z
            \end{circuitdiagram}
            \caption{AND4 gate}
    \end{figure}
As shown in  \autoref{TTL_logic_levels} our acceptable voltages for our CMOS "ON" state range from 2V to VDD (which typically is around $5V\pm0.25V$)

\begin{equation}
V_{on} \geq 2V
\end{equation}

Because there are four transistors dividing the voltage by being in series, the power supply voltage is being divided by the amount of transistors in series.
In order to match the threshold voltages of all of the transistors, which is needed for a working digital logic, the following equation need to be satisfied

\begin{equation}
V_{on} > 4 \cdot V_{Tn}
\end{equation}

Lets assume the worst case with

\begin{equation}
V_{on} = 2V
\end{equation}

Which leads to the required worst case threshold tolerance value

\begin{equation}
4 \cdot V_{Tn} < 2V
\Rightarrow
V_{Tn} < 500mV
\end{equation}
 
Using the equation from \autoref{transistor_threshold_calculation} we can now set in this value and play around with the parameters.
  

 
 \begin{equation}
V_T= \frac{1.66 \cdot 10^{-7}}{C_{ox}}-0.094
\end{equation}

For safety reasons we take 10\% of the worst case as a error margin which leads us to the following new equation to solve
 \begin{equation}
V_T = 450mV
\end{equation}

This gives us the constraint of
\begin{equation}
0.45 V = \left( \frac{1.66 \cdot 10^{-7}}{C_{ox}}-0.094 \right) V
\end{equation}

We solve for the target capacity (in $\frac{F}{cm^2}$) which gives us the desired threshold voltage:
\begin{equation}
\Rightarrow
C_{ox}
=
3.057 \cdot {10}^{-7} \frac{F}{cm^2}
\end{equation}
\begin{equation}
\Rightarrow
C_{ox}
=
305.7 \frac{nF}{cm^2}
\end{equation}
