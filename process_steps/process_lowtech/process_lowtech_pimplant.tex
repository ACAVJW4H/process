\section{p+ Implant}\label{pimplant}
For the bulk of the NMOS transistors and for the source and drain of the PMOS transistors highly doped  p+ areas are required.
In this step we're going to build these.

\begin{figure}[H]
	\centering
	\begin{tikzpicture}[node distance = 3cm, auto, thick,scale=\CrossAndTopSectionBig, every node/.style={transform shape}]
		\input{tikz_process_steps/sti.a.tex}
% n-well
\fill[nwell] (1,0.75) rectangle (8.5,2);
\node at (5.75,1) {N-Well};
\fill[nimplant] (1.5,1) rectangle (3,2);
\node at (2,1.5) {n+};
\fill[nimplant] (15,1) rectangle (16.5,2);
\node at (16,1.5) {n+};
\fill[nimplant] (12,1) rectangle (13.5,2);
\node at (13,1.5) {n+};
\fill[pimplant] (3.0,1.5) rectangle (5,2);
\node at (4,1.65) {p+};
\fill[pimplant] (6.5,1.5) rectangle (8.5,2);
\node at (7,1.65) {p+};
\fill[pimplant] (17,1.5) rectangle (18.75,2);
\node at (18,1.65) {p+};
	\end{tikzpicture}
	\begin{tikzpicture}[node distance = 3cm, auto, thick,scale=\CrossAndTopSectionBig, every node/.style={transform shape}]
		\fill[isolationoxide] (0,0) rectangle (20,10);

% n-well
\fill[nwell] (1,1.25) rectangle (8.5,7.5);

% p-well
\fill[pwell] (11.5,1.25) rectangle (19,7.5);

% gate metal
\fill[gatemetal] (5,0) rectangle (6.5,9);
\fill[gatemetal] (13.5,0) rectangle (15,9);
\fill[gatemetal] (5,8) rectangle (15,10);

% n+
\fill[nimplant] (1.5,2) rectangle (3,6.5);
\fill[nimplant] (12,2) rectangle (13.5,6.5);
\fill[nimplant] (15,2) rectangle (16.5,6.5);

% p+
\fill[pimplant] (3.5,2) rectangle (5,6.5);
\fill[pimplant] (6.5,2) rectangle (8,6.5);
\fill[pimplant] (17,2) rectangle (18.5,6.5);

	\end{tikzpicture}
	\caption{P+ implant geometry target}
\end{figure}

The tricky thing here is to have a reasonable implant depth but not too deep because the deeper the junction, the higher the junction capacity which in turn limits the switching performance of the CMOS circuitry.

\begin{figure}[H]
	\centering
	\begin{tikzpicture}[node distance =1cm, auto, thick,scale=\VLSILayout, every node/.style={transform shape}]
		\input{tikz_process_steps/pimplant.layout.tex}
% gate metal
\fill[gatemetal,opacity=\OpacityLayout] (4.8,1.75) rectangle (6.7,8);
\fill[gatemetal,opacity=\OpacityLayout] (13.3,1.75) rectangle (15.2,8);
\fill[gatemetal,opacity=\OpacityLayout] (4.8,8) rectangle (15.2,10);


% n+
\fill[nimplant,opacity=\OpacityLayout] (1.5,2) rectangle (3,6.5);
\fill[nimplant,opacity=\OpacityLayout] (12,2) rectangle (16.5,6.5);


% p+
\fill[pimplant,opacity=\OpacityLayout] (3,0.75) rectangle (8.5,7.5);
\fill[pimplant,opacity=\OpacityLayout] (17,0.75) rectangle (19,7.5);
	\end{tikzpicture}
	\caption{P+ layout}
	\label{pimplant_layout}
\end{figure}

An example layout of p-implants can be seen in \autoref{pimplant_layout}, the mask is being extracted from the layer "p\_plus\_select"

Also important to notice is that this example layout is just for demonstration purposes only, please have a look at the standard cell documentation for the actual layouts. 

\newpage

\subsection{Hard mask: Dioxide growth}

In order to selectively inject charge carrying atoms into the crystalline structure a protective dioxide ($SiO_2$) layer needs to be grown on top of a p-type substrate.

\begin{figure}[H]
	\centering
	\begin{tikzpicture}[node distance = 3cm, auto, thick,scale=\CrossSectionOnly, every node/.style={transform shape}]
		\input{tikz_process_steps/sti.a.tex}
% n-well
\fill[nwell] (1,0.75) rectangle (8.5,2);
\node at (5.75,1) {N-Well};
\fill[nimplant] (1.5,1) rectangle (3,2);
\node at (2,1.5) {n+};
\fill[nimplant] (15,1) rectangle (16.5,2);
\node at (16,1.5) {n+};
\fill[nimplant] (12,1) rectangle (13.5,2);
\node at (13,1.5) {n+};
	\end{tikzpicture}
	\drawStepArrow{Wet oxidation}
	\begin{tikzpicture}[node distance = 3cm, auto, thick,scale=\CrossSectionOnly, every node/.style={transform shape}]
		% oxide
\fill[isolationoxide] (0,2) rectangle (20,3.5);

\input{tikz_process_steps/sti.a.tex}
% n-well
\fill[nwell] (1,0.75) rectangle (8.5,2);
\node at (5.75,1) {N-Well};
\fill[nimplant] (1.5,1) rectangle (3,2);
\node at (2,1.5) {n+};
\fill[nimplant] (15,1) rectangle (16.5,2);
\node at (16,1.5) {n+};
\fill[nimplant] (12,1) rectangle (13.5,2);
\node at (13,1.5) {n+};
	\end{tikzpicture}
	\caption{Oxide layer}
\end{figure}

With an energy of 13keV for the implantation performed in \autoref{nimplant_implant_step}, the projected range of the dopants within the oxide will be 40nm (58nm tops) \footnote{\url{http://cleanroom.byu.edu/rangestraggle}}.
This means being on the safe side and having 100nm as the thickness is a good approach.
In order to grow the 100nm thick oxide layer, the wafer is being oxidized for around 5 minutes 30 seconds at 1050\degree C using wet oxidation which results in a dioxide layer of around 100nm in thickness\footnote{\url{http://cleanroom.byu.edu/OxideTimeCalc}}.

\subsection{Hard mask: Patterning}

The resist is being deposited using spin coating and then baked depending on the baking time for the specific resist.
The layout for being exposed onto the resist is being extracted from the "p\_plus\_select" layer within the GDS2 file onto a \textbf{bright field} mask.
The requirement is a \textbf{negative} tone resist.

\begin{figure}[H]
	\centering
	\begin{tikzpicture}[node distance = 3cm, auto, thick,scale=\CrossAndTopSection, every node/.style={transform shape}]
		% oxide
\fill[isolationoxide] (0,2) rectangle (20,3.5);

\input{tikz_process_steps/sti.a.tex}
% n-well
\fill[nwell] (1,0.75) rectangle (8.5,2);
\node at (5.75,1) {N-Well};
\fill[nimplant] (1.5,1) rectangle (3,2);
\node at (2,1.5) {n+};
\fill[nimplant] (15,1) rectangle (16.5,2);
\node at (16,1.5) {n+};
\fill[nimplant] (12,1) rectangle (13.5,2);
\node at (13,1.5) {n+};
	\end{tikzpicture}
	\begin{tikzpicture}[node distance = 3cm, auto, thick,scale=\CrossAndTopSection, every node/.style={transform shape}]
		\fill[isolationoxide] (0,0) rectangle (20,12);
	\end{tikzpicture}
	\drawStepArrow{Mask: pselect}
	\begin{tikzpicture}[node distance = 3cm, auto, thick,scale=\CrossAndTopSection, every node/.style={transform shape}]
		% resist
\fill[resist] (0,2.0) rectangle (3.75,5.0);
\fill[resist] (8.75,2.0) rectangle (16.75,5.0);
\fill[resist] (19.25,2.0) rectangle (20,5.0);

% oxide
\fill[isolationoxide] (0,2) rectangle (20,3.5);

\input{tikz_process_steps/well.a.tex}
\fill[nimplant] (1.5,1) rectangle (3,2);
\node at (2,1.5) {n+};
\fill[nimplant] (15,1) rectangle (16.5,2);
\node at (16,1.5) {n+};
\fill[nimplant] (12,1) rectangle (13.5,2);
\node at (13,1.5) {n+};

	\end{tikzpicture}
	\begin{tikzpicture}[node distance = 3cm, auto, thick,scale=\CrossAndTopSection, every node/.style={transform shape}]
		\fill[resist] (0,0) rectangle (20,12);

% p+
\fill[isolationoxide] (3.5,2) rectangle (5,6.5);
\fill[isolationoxide] (6.5,2) rectangle (8,6.5);
\fill[isolationoxide] (17,2) rectangle (18.5,6.5);
	\end{tikzpicture}
	\caption{P+ region resist mask}
\end{figure}

The thickness of the resist layer and the baking duration will variate depending on the specific equipment for which this process will be implemented with.
Also after the exposure and development, the hard baking shouldn't be forgotten!

\newpage

\subsection{Hard mask: Etching}

We now need to open a window in the dioxide layer, through which we will inject carrier atoms into the silicon crystal structure.

\begin{figure}[H]
	\centering
	\begin{tikzpicture}[node distance = 3cm, auto, thick,scale=\CrossAndTopSection, every node/.style={transform shape}]
		% oxide
\fill[isolationoxide] (0,2) rectangle (20,3.5);

% resist
\fill[resist] (0,3.5) rectangle (3.0,4.1);
\fill[resist] (8.5,3.5) rectangle (17,4.1);
\fill[resist] (18.75,3.5) rectangle (20,4.1);

\input{tikz_process_steps/sti.a.tex}
% n-well
\fill[nwell] (1,0.75) rectangle (8.5,2);
\node at (5.75,1) {N-Well};
\fill[nimplant] (1.5,1) rectangle (3,2);
\node at (2,1.5) {n+};
\fill[nimplant] (15,1) rectangle (16.5,2);
\node at (16,1.5) {n+};
\fill[nimplant] (12,1) rectangle (13.5,2);
\node at (13,1.5) {n+};
	\end{tikzpicture}
	\begin{tikzpicture}[node distance = 3cm, auto, thick,scale=\CrossAndTopSection, every node/.style={transform shape}]
		\fill[resist] (0,0) rectangle (20,12);

% p+
\fill[isolationoxide] (3.5,2) rectangle (5,6.5);
\fill[isolationoxide] (6.5,2) rectangle (8,6.5);
\fill[isolationoxide] (17,2) rectangle (18.5,6.5);
	\end{tikzpicture}
	\drawStepArrow{}
	\begin{tikzpicture}[node distance = 3cm, auto, thick,scale=\CrossAndTopSection, every node/.style={transform shape}]
		% resist
\fill[resist] (0,3.5) rectangle (3.0,6.0);
\fill[resist] (8.25,3.5) rectangle (17,6.0);
\fill[resist] (18.75,3.5) rectangle (20,6.0);

\fill[isolationoxide] (0,4.0) rectangle (0.5,4.75);
\fill[isolationoxide] (9.00,4.0) rectangle (11.0,4.75);
\fill[isolationoxide] (19.5,4.0) rectangle (20,4.75);

\filldraw[line width=0, isolationoxide] (0.5,4.75) -- (0.5,4.0) -- (1.25,4.0);
\filldraw[line width=0, isolationoxide] (8.25,4.0) -- (9.00,4.0) -- (9.00,4.75);

\filldraw[line width=0, isolationoxide] (11.0,4.75) -- (11.0,4.0) -- (11.75,4.0);
\filldraw[line width=0, isolationoxide] (18.75,4.0) -- (19.5,4.0) -- (19.5,4.75);

% oxide
\fill[isolationoxide] (0,2) rectangle (3.0,4.0);
\fill[isolationoxide] (8.25,2) rectangle (17,4.0);
\fill[isolationoxide] (18.75,2) rectangle (20,4.0);

\input{tikz_process_steps/well.a.tex}
\fill[nimplant] (1.5,1) rectangle (3,2);
\node at (2,1.5) {n+};
\fill[nimplant] (15,1) rectangle (16.5,2);
\node at (16,1.5) {n+};
\fill[nimplant] (12,1) rectangle (13.5,2);
\node at (13,1.5) {n+};

	\end{tikzpicture}
	\begin{tikzpicture}[node distance = 3cm, auto, thick,scale=\CrossAndTopSection, every node/.style={transform shape}]
		\fill[resist] (0,0) rectangle (20,12);

% p+
\fill[nwell] (3.5,2) rectangle (5,6.5);
\fill[nwell] (6.5,2) rectangle (8,6.5);
\fill[substrate] (17,2) rectangle (18.25,6.5);
	\end{tikzpicture}
	\caption{P+ region opened}
\end{figure}

Since the silicon dioxide layer is 100nm thick and we wanna reach the silicon below we can use wet etching as described in the chemistry chapter.\\

\textbf{Possible approaches}:
\begin{itemize}
	\item \textbf{"AOE Etcher (DRY-AOE)" from HKUST} \\
	We can use anisotropic plasma etching for sharper borders.
	\item \textbf{Chemical solution} \\
	We can use buffered hydrofluoric acid (BOE (1:6)) at room temperature for one minutes in order to get through the 100nm of oxide.\\
	(Too long over 1 minutes might cause under-etch however!)
\end{itemize}

\subsection{Hard mask: Resist strip}

Now we need to remove the contaminants for further processing.

\begin{figure}[H]
	\centering
	\begin{tikzpicture}[node distance = 3cm, auto, thick,scale=\CrossSectionOnly, every node/.style={transform shape}]
		% oxide
\fill[isolationoxide] (0,2) rectangle (3.0,3.5);
\fill[isolationoxide] (8.5,2) rectangle (17,3.5);
\fill[isolationoxide] (18.75,2) rectangle (20,3.5);

% resist
\fill[resist] (0,3.5) rectangle (3.0,4.1);
\fill[resist] (8.5,3.5) rectangle (17,4.1);
\fill[resist] (18.75,3.5) rectangle (20,4.1);

\input{tikz_process_steps/sti.a.tex}
% n-well
\fill[nwell] (1,0.75) rectangle (8.5,2);
\node at (5.75,1) {N-Well};
\fill[nimplant] (1.5,1) rectangle (3,2);
\node at (2,1.5) {n+};
\fill[nimplant] (15,1) rectangle (16.5,2);
\node at (16,1.5) {n+};
\fill[nimplant] (12,1) rectangle (13.5,2);
\node at (13,1.5) {n+};
	\end{tikzpicture}
	\drawStepArrow{}
	\begin{tikzpicture}[node distance = 3cm, auto, thick,scale=\CrossSectionOnly, every node/.style={transform shape}]
		\fill[isolationoxide] (0,4.0) rectangle (0.5,4.75);
\fill[isolationoxide] (9.00,4.0) rectangle (11.0,4.75);
\fill[isolationoxide] (19.5,4.0) rectangle (20,4.75);

\filldraw[line width=0, isolationoxide] (0.5,4.75) -- (0.5,4.0) -- (1.25,4.0);
\filldraw[line width=0, isolationoxide] (8.25,4.0) -- (9.00,4.0) -- (9.00,4.75);

\filldraw[line width=0, isolationoxide] (11.0,4.75) -- (11.0,4.0) -- (11.75,4.0);
\filldraw[line width=0, isolationoxide] (18.75,4.0) -- (19.5,4.0) -- (19.5,4.75);

% oxide
\fill[isolationoxide] (0,2) rectangle (3.0,4.0);
\fill[isolationoxide] (8.25,2) rectangle (17,4.0);
\fill[isolationoxide] (18.75,2) rectangle (20,4.0);

\input{tikz_process_steps/sti.a.tex}
% n-well
\fill[nwell] (1,0.75) rectangle (8.5,2);
\node at (5.75,1) {N-Well};
\fill[nimplant] (1.5,1) rectangle (3,2);
\node at (2,1.5) {n+};
\fill[nimplant] (15,1) rectangle (16.5,2);
\node at (16,1.5) {n+};
\fill[nimplant] (12,1) rectangle (13.5,2);
\node at (13,1.5) {n+};
	\end{tikzpicture}
	\caption{Resist removal}
\end{figure}

We strip the resist, rinse and perform sulfuric cleaning.

\newpage

\subsection{Implantation/Doping}\label{pimplant_implant_step}

We now need to bring in the carriers in order to build the p-junctions.

\begin{figure}[H]
	\centering
	\begin{minipage}{0.5\textwidth}
	\centering
	\begin{tikzpicture}[node distance = 3cm, auto, thick,scale=\CrossSectionOnly, every node/.style={transform shape}]
		\forloop{ct}{0}{\value{ct} < 21}
{
	\draw [->] (\value{ct},6) -- (\value{ct},5);
	\node at (\value{ct},6.2) {B$^{11}$};
}

\fill[isolationoxide] (0,4.0) rectangle (0.5,4.75);
\fill[isolationoxide] (9.00,4.0) rectangle (11.0,4.75);
\fill[isolationoxide] (19.5,4.0) rectangle (20,4.75);

\filldraw[line width=0, isolationoxide] (0.5,4.75) -- (0.5,4.0) -- (1.25,4.0);
\filldraw[line width=0, isolationoxide] (8.25,4.0) -- (9.00,4.0) -- (9.00,4.75);

\filldraw[line width=0, isolationoxide] (11.0,4.75) -- (11.0,4.0) -- (11.75,4.0);
\filldraw[line width=0, isolationoxide] (18.75,4.0) -- (19.5,4.0) -- (19.5,4.75);

% oxide
\fill[isolationoxide] (0,2) rectangle (3.0,4.0);
\fill[isolationoxide] (8.25,2) rectangle (17,4.0);
\fill[isolationoxide] (18.75,2) rectangle (20,4.0);

\input{tikz_process_steps/well.a.tex}
\fill[nimplant] (1.5,1) rectangle (3,2);
\node at (2,1.5) {n+};
\fill[nimplant] (15,1) rectangle (16.5,2);
\node at (16,1.5) {n+};
\fill[nimplant] (12,1) rectangle (13.5,2);
\node at (13,1.5) {n+};

	\end{tikzpicture}
	\drawStepArrow{Boron implant}
	\begin{tikzpicture}[node distance = 3cm, auto, thick,scale=\CrossSectionOnly, every node/.style={transform shape}]
		\fill[isolationoxide] (0,4.0) rectangle (0.5,4.75);
\fill[isolationoxide] (9.00,4.0) rectangle (11.0,4.75);
\fill[isolationoxide] (19.5,4.0) rectangle (20,4.75);

\filldraw[line width=0, isolationoxide] (0.5,4.75) -- (0.5,4.0) -- (1.25,4.0);
\filldraw[line width=0, isolationoxide] (8.25,4.0) -- (9.00,4.0) -- (9.00,4.75);

\filldraw[line width=0, isolationoxide] (11.0,4.75) -- (11.0,4.0) -- (11.75,4.0);
\filldraw[line width=0, isolationoxide] (18.75,4.0) -- (19.5,4.0) -- (19.5,4.75);

% oxide
\fill[isolationoxide] (0,2) rectangle (3.0,4.0);
\fill[isolationoxide] (8.25,2) rectangle (17,4.0);
\fill[isolationoxide] (18.75,2) rectangle (20,4.0);

\input{tikz_process_steps/well.a.tex}
\fill[nimplant] (1.5,1) rectangle (3,2);
\node at (2,1.5) {n+};
\fill[nimplant] (15,1) rectangle (16.5,2);
\node at (16,1.5) {n+};
\fill[nimplant] (12,1) rectangle (13.5,2);
\node at (13,1.5) {n+};
\shade[upper left = pimplant, upper right = pimplant, lower right = nwell, lower left = nwell,]  (3.0,1.75) rectangle (5,2);
\shade[upper left = pimplant, upper right = pimplant, lower right = nwell, lower left = nwell,]  (6.5,1.75) rectangle (8.25,2);
\shade[upper left = pimplant, upper right = pimplant, lower right = pwell, lower left = pwell,]  (17,1.75) rectangle (18.75,2);
	\end{tikzpicture}
	\drawStepArrow{Annealing}
	\begin{tikzpicture}[node distance = 3cm, auto, thick,scale=\CrossSectionOnly, every node/.style={transform shape}]
		\fill[isolationoxide] (0,4.0) rectangle (0.5,4.75);
\fill[isolationoxide] (9.00,4.0) rectangle (11.0,4.75);
\fill[isolationoxide] (19.5,4.0) rectangle (20,4.75);

\filldraw[line width=0, isolationoxide] (0.5,4.75) -- (0.5,4.0) -- (1.25,4.0);
\filldraw[line width=0, isolationoxide] (8.25,4.0) -- (9.00,4.0) -- (9.00,4.75);

\filldraw[line width=0, isolationoxide] (11.0,4.75) -- (11.0,4.0) -- (11.75,4.0);
\filldraw[line width=0, isolationoxide] (18.75,4.0) -- (19.5,4.0) -- (19.5,4.75);

% oxide
\fill[isolationoxide] (0,2) rectangle (3.0,4.0);
\fill[isolationoxide] (8.25,2) rectangle (17,4.0);
\fill[isolationoxide] (18.75,2) rectangle (20,4.0);

\input{tikz_process_steps/well.a.tex}
\fill[nimplant] (1.5,1) rectangle (3,2);
\node at (2,1.5) {n+};
\fill[nimplant] (15,1) rectangle (16.5,2);
\node at (16,1.5) {n+};
\fill[nimplant] (12,1) rectangle (13.5,2);
\node at (13,1.5) {n+};

% after anneal of p implant:
\shade[upper left = nimplant, upper right = nimplant, lower right = nwell, lower left = nwell,] (1.25,1.25) rectangle (3,2);
\node at (2,1.65) {n+};
\shade[upper left = nimplant, upper right = nimplant, lower right = pwell, lower left = pwell,] (11.75,1.25) rectangle (13.5,2);
\node at (12.5,1.65) {n+};
\shade[upper left = nimplant, upper right = nimplant, lower right = pwell, lower left = pwell,] (15,1.25) rectangle (17,2);
\node at (16,1.65) {n+};

% p implant after anneal:
\shade[upper left = pimplant, upper right = pimplant, lower right = nwell, lower left = nwell,] (3.0,1.5) rectangle (5,2);
\node at (4,1.65) {p+};
\shade[upper left = pimplant, upper right = pimplant, lower right = nwell, lower left = nwell,] (6.5,1.5) rectangle (8.5,2);
\node at (7,1.65) {p+};
\shade[upper left = pimplant, upper right = pimplant, lower right = pwell, lower left = pwell,] (17,1.5) rectangle (18.75,2);
\node at (18,1.65) {p+};
	\end{tikzpicture} \\
	\textbf{Implantation approach}
	\end{minipage}\begin{minipage}{0.5\textwidth}
	\centering
	\begin{tikzpicture}[node distance = 3cm, auto, thick,scale=\CrossSectionOnly, every node/.style={transform shape}]
		\fill[pimplant] (0,0) rectangle (20,5);

\fill[isolationoxide] (0,4.0) rectangle (0.5,4.75);
\fill[isolationoxide] (9.00,4.0) rectangle (11.0,4.75);
\fill[isolationoxide] (19.5,4.0) rectangle (20,4.75);

\filldraw[line width=0, isolationoxide] (0.5,4.75) -- (0.5,4.0) -- (1.25,4.0);
\filldraw[line width=0, isolationoxide] (8.25,4.0) -- (9.00,4.0) -- (9.00,4.75);

\filldraw[line width=0, isolationoxide] (11.0,4.75) -- (11.0,4.0) -- (11.75,4.0);
\filldraw[line width=0, isolationoxide] (18.75,4.0) -- (19.5,4.0) -- (19.5,4.75);

% oxide
\fill[isolationoxide] (0,2) rectangle (3.0,4.0);
\fill[isolationoxide] (8.25,2) rectangle (17,4.0);
\fill[isolationoxide] (18.75,2) rectangle (20,4.0);

\input{tikz_process_steps/well.a.tex}
\fill[nimplant] (1.5,1) rectangle (3,2);
\node at (2,1.5) {n+};
\fill[nimplant] (15,1) rectangle (16.5,2);
\node at (16,1.5) {n+};
\fill[nimplant] (12,1) rectangle (13.5,2);
\node at (13,1.5) {n+};

	\end{tikzpicture}
	\drawStepArrow{Constant source diffusion}
	\begin{tikzpicture}[node distance = 3cm, auto, thick,scale=\CrossSectionOnly, every node/.style={transform shape}]
		\fill[pimplant] (0,0) rectangle (20,5);

\fill[isolationoxide] (0,4.0) rectangle (0.5,4.75);
\fill[isolationoxide] (9.00,4.0) rectangle (11.0,4.75);
\fill[isolationoxide] (19.5,4.0) rectangle (20,4.75);

\filldraw[line width=0, isolationoxide] (0.5,4.75) -- (0.5,4.0) -- (1.25,4.0);
\filldraw[line width=0, isolationoxide] (8.25,4.0) -- (9.00,4.0) -- (9.00,4.75);

\filldraw[line width=0, isolationoxide] (11.0,4.75) -- (11.0,4.0) -- (11.75,4.0);
\filldraw[line width=0, isolationoxide] (18.75,4.0) -- (19.5,4.0) -- (19.5,4.75);

% oxide
\fill[isolationoxide] (0,2) rectangle (3.0,4.0);
\fill[isolationoxide] (8.25,2) rectangle (17,4.0);
\fill[isolationoxide] (18.75,2) rectangle (20,4.0);

\input{tikz_process_steps/well.a.tex}
\fill[nimplant] (1.5,1) rectangle (3,2);
\node at (2,1.5) {n+};
\fill[nimplant] (15,1) rectangle (16.5,2);
\node at (16,1.5) {n+};
\fill[nimplant] (12,1) rectangle (13.5,2);
\node at (13,1.5) {n+};

% after anneal of p implant:
\shade[upper left = nimplant, upper right = nimplant, lower right = nwell, lower left = nwell,] (1.25,1.25) rectangle (3,2);
\node at (2,1.65) {n+};
\shade[upper left = nimplant, upper right = nimplant, lower right = pwell, lower left = pwell,] (11.75,1.25) rectangle (13.5,2);
\node at (12.5,1.65) {n+};
\shade[upper left = nimplant, upper right = nimplant, lower right = pwell, lower left = pwell,] (15,1.25) rectangle (17,2);
\node at (16,1.65) {n+};

% p implant after anneal:
\shade[upper left = pimplant, upper right = pimplant, lower right = nwell, lower left = nwell,] (3.0,1.5) rectangle (5,2);
\node at (4,1.65) {p+};
\shade[upper left = pimplant, upper right = pimplant, lower right = nwell, lower left = nwell,] (6.5,1.5) rectangle (8.25,2);
\node at (7,1.65) {p+};
\shade[upper left = pimplant, upper right = pimplant, lower right = pwell, lower left = pwell,] (17,1.5) rectangle (18.75,2);
\node at (18,1.65) {p+};
	\end{tikzpicture}
	\drawStepArrow{Source removal}
	\begin{tikzpicture}[node distance = 3cm, auto, thick,scale=\CrossSectionOnly, every node/.style={transform shape}]
		\fill[isolationoxide] (0,4.0) rectangle (0.5,4.75);
\fill[isolationoxide] (9.00,4.0) rectangle (11.0,4.75);
\fill[isolationoxide] (19.5,4.0) rectangle (20,4.75);

\filldraw[line width=0, isolationoxide] (0.5,4.75) -- (0.5,4.0) -- (1.25,4.0);
\filldraw[line width=0, isolationoxide] (8.25,4.0) -- (9.00,4.0) -- (9.00,4.75);

\filldraw[line width=0, isolationoxide] (11.0,4.75) -- (11.0,4.0) -- (11.75,4.0);
\filldraw[line width=0, isolationoxide] (18.75,4.0) -- (19.5,4.0) -- (19.5,4.75);

% oxide
\fill[isolationoxide] (0,2) rectangle (3.0,4.0);
\fill[isolationoxide] (8.25,2) rectangle (17,4.0);
\fill[isolationoxide] (18.75,2) rectangle (20,4.0);

\input{tikz_process_steps/well.a.tex}
\fill[nimplant] (1.5,1) rectangle (3,2);
\node at (2,1.5) {n+};
\fill[nimplant] (15,1) rectangle (16.5,2);
\node at (16,1.5) {n+};
\fill[nimplant] (12,1) rectangle (13.5,2);
\node at (13,1.5) {n+};

% after anneal of p implant:
\shade[upper left = nimplant, upper right = nimplant, lower right = nwell, lower left = nwell,] (1.25,1.25) rectangle (3,2);
\node at (2,1.65) {n+};
\shade[upper left = nimplant, upper right = nimplant, lower right = pwell, lower left = pwell,] (11.75,1.25) rectangle (13.5,2);
\node at (12.5,1.65) {n+};
\shade[upper left = nimplant, upper right = nimplant, lower right = pwell, lower left = pwell,] (15,1.25) rectangle (17,2);
\node at (16,1.65) {n+};

% p implant after anneal:
\shade[upper left = pimplant, upper right = pimplant, lower right = nwell, lower left = nwell,] (3.0,1.5) rectangle (5,2);
\node at (4,1.65) {p+};
\shade[upper left = pimplant, upper right = pimplant, lower right = nwell, lower left = nwell,] (6.5,1.5) rectangle (8.5,2);
\node at (7,1.65) {p+};
\shade[upper left = pimplant, upper right = pimplant, lower right = pwell, lower left = pwell,] (17,1.5) rectangle (18.75,2);
\node at (18,1.65) {p+};
	\end{tikzpicture} \\
	\textbf{Diffusion approach}
	\end{minipage}
	\caption{P+ doping process}
\end{figure}

\textbf{Possible approaches}:
\begin{itemize}
	\item \textbf{"CF-3000 Implanter (IMP-3000)" from HKUST} \\
	At HKUST we have an implanter which gives us better control over the initial surface concentration. \\
	These steps are needed to arrive with the desired geometry:
	\begin{enumerate}
		\item The pselect is implanted with a Boron ($B^{11}$) dose of $2.5\times10^{12}cm^{-2}$ at an energy of 13 keV  (43nm$\pm$18nm deep)
		\item The pselect is annealed for 10 minutes at 1050\degreesC in $N_2$ environment (DIF-A1)
	\end{enumerate}
	\item \textbf{Constant source diffusion} \\
	We can add a layer of Phosphorus solution and diffusing in order to have an initial concentration in order to reach the desired concentration later by main diffusion.
		\begin{enumerate}
		\item A constant source is added (gas or liquid)
		\item The source dopant is driven in for 5 minutes at 1050\degreesC
		\item The dopant source is removed by stopping the gas flow or cleaning the surface
	\end{enumerate}
\end{itemize}

\subsection{Hard mask: Removal}

Now we want to remove the silicon mask from the wafer and clean it for another clean oxide mask layer.

\begin{figure}[H]
	\centering
	\begin{tikzpicture}[node distance = 3cm, auto, thick,scale=\CrossSectionOnly, every node/.style={transform shape}]
		% oxide
\fill[isolationoxide] (0,2) rectangle (3.0,3.5);
\fill[isolationoxide] (8.5,2) rectangle (17,3.5);
\fill[isolationoxide] (18.75,2) rectangle (20,3.5);

\input{tikz_process_steps/sti.a.tex}
% n-well
\fill[nwell] (1,0.75) rectangle (8.5,2);
\node at (5.75,1) {N-Well};
\fill[nimplant] (1.5,1) rectangle (3,2);
\node at (2,1.5) {n+};
\fill[nimplant] (15,1) rectangle (16.5,2);
\node at (16,1.5) {n+};
\fill[nimplant] (12,1) rectangle (13.5,2);
\node at (13,1.5) {n+};

\fill[pimplant] (3.0,1.5) rectangle (5,2);
\fill[pimplant] (6.5,1.5) rectangle (8.5,2);
\fill[pimplant] (17,1.5) rectangle (18.75,2);
	\end{tikzpicture}
	\drawStepArrow{}
	\begin{tikzpicture}[node distance = 3cm, auto, thick,scale=\CrossSectionOnly, every node/.style={transform shape}]
		\input{tikz_process_steps/sti.a.tex}
% n-well
\fill[nwell] (1,0.75) rectangle (8.5,2);
\node at (5.75,1) {N-Well};
\fill[nimplant] (1.5,1) rectangle (3,2);
\node at (2,1.5) {n+};
\fill[nimplant] (15,1) rectangle (16.5,2);
\node at (16,1.5) {n+};
\fill[nimplant] (12,1) rectangle (13.5,2);
\node at (13,1.5) {n+};

\fill[pimplant] (3.0,1.5) rectangle (5,2);
\fill[pimplant] (6.5,1.5) rectangle (8.5,2);
\fill[pimplant] (17,1.5) rectangle (18.75,2);
	\end{tikzpicture}
	\caption{Oxide removal}
\end{figure}

We use buffered hydrofluoric acid (BOE (1:6)) at room temperature for 1 minute in order to remove the 100nm of oxide layer.
