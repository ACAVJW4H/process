\section{p+ Implant}\label{pimplant}
For the bulk of the NMOS transistors and for the source and drain of the PMOS transistors highly doped  p+ areas are required.
In this step we're going to build these.

\begin{figure}[H]
	\centering
	\begin{tikzpicture}[node distance = 3cm, auto, thick,scale=\CrossAndTopSectionBig, every node/.style={transform shape}]
		\input{tikz_process_steps/sti.a.tex}
% n-well
\fill[nwell] (1,0.75) rectangle (8.5,2);
\node at (5.75,1) {N-Well};
\fill[nimplant] (1.5,1) rectangle (3,2);
\node at (2,1.5) {n+};
\fill[nimplant] (15,1) rectangle (16.5,2);
\node at (16,1.5) {n+};
\fill[nimplant] (12,1) rectangle (13.5,2);
\node at (13,1.5) {n+};
\fill[pimplant] (3.0,1.5) rectangle (5,2);
\node at (4,1.65) {p+};
\fill[pimplant] (6.5,1.5) rectangle (8.5,2);
\node at (7,1.65) {p+};
\fill[pimplant] (17,1.5) rectangle (18.75,2);
\node at (18,1.65) {p+};
	\end{tikzpicture}
	\begin{tikzpicture}[node distance = 3cm, auto, thick,scale=\CrossAndTopSectionBig, every node/.style={transform shape}]
		\fill[isolationoxide] (0,0) rectangle (20,10);

% n-well
\fill[nwell] (1,1.25) rectangle (8.5,7.5);

% p-well
\fill[pwell] (11.5,1.25) rectangle (19,7.5);

% gate metal
\fill[gatemetal] (5,0) rectangle (6.5,9);
\fill[gatemetal] (13.5,0) rectangle (15,9);
\fill[gatemetal] (5,8) rectangle (15,10);

% n+
\fill[nimplant] (1.5,2) rectangle (3,6.5);
\fill[nimplant] (12,2) rectangle (13.5,6.5);
\fill[nimplant] (15,2) rectangle (16.5,6.5);

% p+
\fill[pimplant] (3.5,2) rectangle (5,6.5);
\fill[pimplant] (6.5,2) rectangle (8,6.5);
\fill[pimplant] (17,2) rectangle (18.5,6.5);

	\end{tikzpicture}
	\caption{P+ implant geometry target}
\end{figure}

The tricky thing here is to have a reasonable implant depth but not too deep because the deeper the junction, the higher the junction capacity which in turn limits the switching performance of the CMOS circuitry.

\begin{figure}[H]
	\centering
	\begin{tikzpicture}[node distance =1cm, auto, thick,scale=\VLSILayout, every node/.style={transform shape}]
		\input{tikz_process_steps/pimplant.layout.tex}
% gate metal
\fill[gatemetal,opacity=\OpacityLayout] (4.8,1.75) rectangle (6.7,8);
\fill[gatemetal,opacity=\OpacityLayout] (13.3,1.75) rectangle (15.2,8);
\fill[gatemetal,opacity=\OpacityLayout] (4.8,8) rectangle (15.2,10);


% n+
\fill[nimplant,opacity=\OpacityLayout] (1.5,2) rectangle (3,6.5);
\fill[nimplant,opacity=\OpacityLayout] (12,2) rectangle (16.5,6.5);


% p+
\fill[pimplant,opacity=\OpacityLayout] (3,0.75) rectangle (8.5,7.5);
\fill[pimplant,opacity=\OpacityLayout] (17,0.75) rectangle (19,7.5);
	\end{tikzpicture}
	\caption{P+ layout}
	\label{pimplant_layout}
\end{figure}

An example layout of p-implants can be seen in \autoref{pimplant_layout}, the mask is being extracted from the layer "p\_plus\_select"

Also important to notice is that this example layout is just for demonstration purposes only, please have a look at the standard cell documentation for the actual layouts. 

\newpage

\subsection{Mask dioxide layer}
\begin{figure}[H]
	\centering
	\begin{tikzpicture}[node distance = 3cm, auto, thick,scale=\CrossSectionOnly, every node/.style={transform shape}]
		\input{tikz_process_steps/sti.a.tex}
% n-well
\fill[nwell] (1,0.75) rectangle (8.5,2);
\node at (5.75,1) {N-Well};
\fill[nimplant] (1.5,1) rectangle (3,2);
\node at (2,1.5) {n+};
\fill[nimplant] (15,1) rectangle (16.5,2);
\node at (16,1.5) {n+};
\fill[nimplant] (12,1) rectangle (13.5,2);
\node at (13,1.5) {n+};
	\end{tikzpicture}
	\drawStepArrow{}
	\begin{tikzpicture}[node distance = 3cm, auto, thick,scale=\CrossSectionOnly, every node/.style={transform shape}]
		% oxide
\fill[isolationoxide] (0,2) rectangle (20,3.5);

\input{tikz_process_steps/sti.a.tex}
% n-well
\fill[nwell] (1,0.75) rectangle (8.5,2);
\node at (5.75,1) {N-Well};
\fill[nimplant] (1.5,1) rectangle (3,2);
\node at (2,1.5) {n+};
\fill[nimplant] (15,1) rectangle (16.5,2);
\node at (16,1.5) {n+};
\fill[nimplant] (12,1) rectangle (13.5,2);
\node at (13,1.5) {n+};
	\end{tikzpicture}
	\caption{Oxide layer}
\end{figure}

\subsection{Pattering}
\begin{figure}[H]
	\centering
	\begin{tikzpicture}[node distance = 3cm, auto, thick,scale=\CrossAndTopSection, every node/.style={transform shape}]
		% oxide
\fill[isolationoxide] (0,2) rectangle (20,3.5);

\input{tikz_process_steps/sti.a.tex}
% n-well
\fill[nwell] (1,0.75) rectangle (8.5,2);
\node at (5.75,1) {N-Well};
\fill[nimplant] (1.5,1) rectangle (3,2);
\node at (2,1.5) {n+};
\fill[nimplant] (15,1) rectangle (16.5,2);
\node at (16,1.5) {n+};
\fill[nimplant] (12,1) rectangle (13.5,2);
\node at (13,1.5) {n+};
	\end{tikzpicture}
	\begin{tikzpicture}[node distance = 3cm, auto, thick,scale=\CrossAndTopSection, every node/.style={transform shape}]
		\fill[isolationoxide] (0,0) rectangle (20,12);
	\end{tikzpicture}
	\drawStepArrow{}
	\begin{tikzpicture}[node distance = 3cm, auto, thick,scale=\CrossAndTopSection, every node/.style={transform shape}]
		% resist
\fill[resist] (0,2.0) rectangle (3.75,5.0);
\fill[resist] (8.75,2.0) rectangle (16.75,5.0);
\fill[resist] (19.25,2.0) rectangle (20,5.0);

% oxide
\fill[isolationoxide] (0,2) rectangle (20,3.5);

\input{tikz_process_steps/well.a.tex}
\fill[nimplant] (1.5,1) rectangle (3,2);
\node at (2,1.5) {n+};
\fill[nimplant] (15,1) rectangle (16.5,2);
\node at (16,1.5) {n+};
\fill[nimplant] (12,1) rectangle (13.5,2);
\node at (13,1.5) {n+};

	\end{tikzpicture}
	\begin{tikzpicture}[node distance = 3cm, auto, thick,scale=\CrossAndTopSection, every node/.style={transform shape}]
		\fill[resist] (0,0) rectangle (20,12);

% p+
\fill[isolationoxide] (3.5,2) rectangle (5,6.5);
\fill[isolationoxide] (6.5,2) rectangle (8,6.5);
\fill[isolationoxide] (17,2) rectangle (18.5,6.5);
	\end{tikzpicture}
	\caption{P+ region resist mask}
\end{figure}

\subsection{Etching}
\begin{figure}[H]
	\centering
	\begin{tikzpicture}[node distance = 3cm, auto, thick,scale=\CrossAndTopSection, every node/.style={transform shape}]
		% oxide
\fill[isolationoxide] (0,2) rectangle (20,3.5);

% resist
\fill[resist] (0,3.5) rectangle (3.0,4.1);
\fill[resist] (8.5,3.5) rectangle (17,4.1);
\fill[resist] (18.75,3.5) rectangle (20,4.1);

\input{tikz_process_steps/sti.a.tex}
% n-well
\fill[nwell] (1,0.75) rectangle (8.5,2);
\node at (5.75,1) {N-Well};
\fill[nimplant] (1.5,1) rectangle (3,2);
\node at (2,1.5) {n+};
\fill[nimplant] (15,1) rectangle (16.5,2);
\node at (16,1.5) {n+};
\fill[nimplant] (12,1) rectangle (13.5,2);
\node at (13,1.5) {n+};
	\end{tikzpicture}
	\begin{tikzpicture}[node distance = 3cm, auto, thick,scale=\CrossAndTopSection, every node/.style={transform shape}]
		\fill[resist] (0,0) rectangle (20,12);

% p+
\fill[isolationoxide] (3.5,2) rectangle (5,6.5);
\fill[isolationoxide] (6.5,2) rectangle (8,6.5);
\fill[isolationoxide] (17,2) rectangle (18.5,6.5);
	\end{tikzpicture}
	\drawStepArrow{}
	\begin{tikzpicture}[node distance = 3cm, auto, thick,scale=\CrossAndTopSection, every node/.style={transform shape}]
		% resist
\fill[resist] (0,3.5) rectangle (3.0,6.0);
\fill[resist] (8.25,3.5) rectangle (17,6.0);
\fill[resist] (18.75,3.5) rectangle (20,6.0);

\fill[isolationoxide] (0,4.0) rectangle (0.5,4.75);
\fill[isolationoxide] (9.00,4.0) rectangle (11.0,4.75);
\fill[isolationoxide] (19.5,4.0) rectangle (20,4.75);

\filldraw[line width=0, isolationoxide] (0.5,4.75) -- (0.5,4.0) -- (1.25,4.0);
\filldraw[line width=0, isolationoxide] (8.25,4.0) -- (9.00,4.0) -- (9.00,4.75);

\filldraw[line width=0, isolationoxide] (11.0,4.75) -- (11.0,4.0) -- (11.75,4.0);
\filldraw[line width=0, isolationoxide] (18.75,4.0) -- (19.5,4.0) -- (19.5,4.75);

% oxide
\fill[isolationoxide] (0,2) rectangle (3.0,4.0);
\fill[isolationoxide] (8.25,2) rectangle (17,4.0);
\fill[isolationoxide] (18.75,2) rectangle (20,4.0);

\input{tikz_process_steps/well.a.tex}
\fill[nimplant] (1.5,1) rectangle (3,2);
\node at (2,1.5) {n+};
\fill[nimplant] (15,1) rectangle (16.5,2);
\node at (16,1.5) {n+};
\fill[nimplant] (12,1) rectangle (13.5,2);
\node at (13,1.5) {n+};

	\end{tikzpicture}
	\begin{tikzpicture}[node distance = 3cm, auto, thick,scale=\CrossAndTopSection, every node/.style={transform shape}]
		\fill[resist] (0,0) rectangle (20,12);

% p+
\fill[nwell] (3.5,2) rectangle (5,6.5);
\fill[nwell] (6.5,2) rectangle (8,6.5);
\fill[substrate] (17,2) rectangle (18.25,6.5);
	\end{tikzpicture}
	\caption{P+ region opened}
\end{figure}

\subsection{Cleaning}
\begin{figure}[H]
	\centering
	\begin{tikzpicture}[node distance = 3cm, auto, thick,scale=\CrossSectionOnly, every node/.style={transform shape}]
		% oxide
\fill[isolationoxide] (0,2) rectangle (3.0,3.5);
\fill[isolationoxide] (8.5,2) rectangle (17,3.5);
\fill[isolationoxide] (18.75,2) rectangle (20,3.5);

% resist
\fill[resist] (0,3.5) rectangle (3.0,4.1);
\fill[resist] (8.5,3.5) rectangle (17,4.1);
\fill[resist] (18.75,3.5) rectangle (20,4.1);

\input{tikz_process_steps/sti.a.tex}
% n-well
\fill[nwell] (1,0.75) rectangle (8.5,2);
\node at (5.75,1) {N-Well};
\fill[nimplant] (1.5,1) rectangle (3,2);
\node at (2,1.5) {n+};
\fill[nimplant] (15,1) rectangle (16.5,2);
\node at (16,1.5) {n+};
\fill[nimplant] (12,1) rectangle (13.5,2);
\node at (13,1.5) {n+};
	\end{tikzpicture}
	\drawStepArrow{}
	\begin{tikzpicture}[node distance = 3cm, auto, thick,scale=\CrossSectionOnly, every node/.style={transform shape}]
		\fill[isolationoxide] (0,4.0) rectangle (0.5,4.75);
\fill[isolationoxide] (9.00,4.0) rectangle (11.0,4.75);
\fill[isolationoxide] (19.5,4.0) rectangle (20,4.75);

\filldraw[line width=0, isolationoxide] (0.5,4.75) -- (0.5,4.0) -- (1.25,4.0);
\filldraw[line width=0, isolationoxide] (8.25,4.0) -- (9.00,4.0) -- (9.00,4.75);

\filldraw[line width=0, isolationoxide] (11.0,4.75) -- (11.0,4.0) -- (11.75,4.0);
\filldraw[line width=0, isolationoxide] (18.75,4.0) -- (19.5,4.0) -- (19.5,4.75);

% oxide
\fill[isolationoxide] (0,2) rectangle (3.0,4.0);
\fill[isolationoxide] (8.25,2) rectangle (17,4.0);
\fill[isolationoxide] (18.75,2) rectangle (20,4.0);

\input{tikz_process_steps/sti.a.tex}
% n-well
\fill[nwell] (1,0.75) rectangle (8.5,2);
\node at (5.75,1) {N-Well};
\fill[nimplant] (1.5,1) rectangle (3,2);
\node at (2,1.5) {n+};
\fill[nimplant] (15,1) rectangle (16.5,2);
\node at (16,1.5) {n+};
\fill[nimplant] (12,1) rectangle (13.5,2);
\node at (13,1.5) {n+};
	\end{tikzpicture}
	\caption{Resist removal}
\end{figure}

\subsection{Implantation/Doping}\label{pimplant_implant_step}
\begin{figure}[H]
	\centering
	\begin{tikzpicture}[node distance = 3cm, auto, thick,scale=\CrossSectionOnly, every node/.style={transform shape}]
		\forloop{ct}{0}{\value{ct} < 21}
{
	\draw [->] (\value{ct},6) -- (\value{ct},5);
	\node at (\value{ct},6.2) {B$^{11}$};
}

\fill[isolationoxide] (0,4.0) rectangle (0.5,4.75);
\fill[isolationoxide] (9.00,4.0) rectangle (11.0,4.75);
\fill[isolationoxide] (19.5,4.0) rectangle (20,4.75);

\filldraw[line width=0, isolationoxide] (0.5,4.75) -- (0.5,4.0) -- (1.25,4.0);
\filldraw[line width=0, isolationoxide] (8.25,4.0) -- (9.00,4.0) -- (9.00,4.75);

\filldraw[line width=0, isolationoxide] (11.0,4.75) -- (11.0,4.0) -- (11.75,4.0);
\filldraw[line width=0, isolationoxide] (18.75,4.0) -- (19.5,4.0) -- (19.5,4.75);

% oxide
\fill[isolationoxide] (0,2) rectangle (3.0,4.0);
\fill[isolationoxide] (8.25,2) rectangle (17,4.0);
\fill[isolationoxide] (18.75,2) rectangle (20,4.0);

\input{tikz_process_steps/well.a.tex}
\fill[nimplant] (1.5,1) rectangle (3,2);
\node at (2,1.5) {n+};
\fill[nimplant] (15,1) rectangle (16.5,2);
\node at (16,1.5) {n+};
\fill[nimplant] (12,1) rectangle (13.5,2);
\node at (13,1.5) {n+};

	\end{tikzpicture}
	\drawStepArrow{}
	\begin{tikzpicture}[node distance = 3cm, auto, thick,scale=\CrossSectionOnly, every node/.style={transform shape}]
		\fill[isolationoxide] (0,4.0) rectangle (0.5,4.75);
\fill[isolationoxide] (9.00,4.0) rectangle (11.0,4.75);
\fill[isolationoxide] (19.5,4.0) rectangle (20,4.75);

\filldraw[line width=0, isolationoxide] (0.5,4.75) -- (0.5,4.0) -- (1.25,4.0);
\filldraw[line width=0, isolationoxide] (8.25,4.0) -- (9.00,4.0) -- (9.00,4.75);

\filldraw[line width=0, isolationoxide] (11.0,4.75) -- (11.0,4.0) -- (11.75,4.0);
\filldraw[line width=0, isolationoxide] (18.75,4.0) -- (19.5,4.0) -- (19.5,4.75);

% oxide
\fill[isolationoxide] (0,2) rectangle (3.0,4.0);
\fill[isolationoxide] (8.25,2) rectangle (17,4.0);
\fill[isolationoxide] (18.75,2) rectangle (20,4.0);

\input{tikz_process_steps/well.a.tex}
\fill[nimplant] (1.5,1) rectangle (3,2);
\node at (2,1.5) {n+};
\fill[nimplant] (15,1) rectangle (16.5,2);
\node at (16,1.5) {n+};
\fill[nimplant] (12,1) rectangle (13.5,2);
\node at (13,1.5) {n+};
\shade[upper left = pimplant, upper right = pimplant, lower right = nwell, lower left = nwell,]  (3.0,1.75) rectangle (5,2);
\shade[upper left = pimplant, upper right = pimplant, lower right = nwell, lower left = nwell,]  (6.5,1.75) rectangle (8.25,2);
\shade[upper left = pimplant, upper right = pimplant, lower right = pwell, lower left = pwell,]  (17,1.75) rectangle (18.75,2);
	\end{tikzpicture}
	\caption{P+ injection process}
\end{figure}

\subsection{Oxide removal}
\begin{figure}[H]
	\centering
	\begin{tikzpicture}[node distance = 3cm, auto, thick,scale=\CrossSectionOnly, every node/.style={transform shape}]
		% oxide
\fill[isolationoxide] (0,2) rectangle (3.0,3.5);
\fill[isolationoxide] (8.5,2) rectangle (17,3.5);
\fill[isolationoxide] (18.75,2) rectangle (20,3.5);

\input{tikz_process_steps/sti.a.tex}
% n-well
\fill[nwell] (1,0.75) rectangle (8.5,2);
\node at (5.75,1) {N-Well};
\fill[nimplant] (1.5,1) rectangle (3,2);
\node at (2,1.5) {n+};
\fill[nimplant] (15,1) rectangle (16.5,2);
\node at (16,1.5) {n+};
\fill[nimplant] (12,1) rectangle (13.5,2);
\node at (13,1.5) {n+};

\fill[pimplant] (3.0,1.5) rectangle (5,2);
\fill[pimplant] (6.5,1.5) rectangle (8.5,2);
\fill[pimplant] (17,1.5) rectangle (18.75,2);
	\end{tikzpicture}
	\drawStepArrow{}
	\begin{tikzpicture}[node distance = 3cm, auto, thick,scale=\CrossSectionOnly, every node/.style={transform shape}]
		\input{tikz_process_steps/sti.a.tex}
% n-well
\fill[nwell] (1,0.75) rectangle (8.5,2);
\node at (5.75,1) {N-Well};
\fill[nimplant] (1.5,1) rectangle (3,2);
\node at (2,1.5) {n+};
\fill[nimplant] (15,1) rectangle (16.5,2);
\node at (16,1.5) {n+};
\fill[nimplant] (12,1) rectangle (13.5,2);
\node at (13,1.5) {n+};

\fill[pimplant] (3.0,1.5) rectangle (5,2);
\fill[pimplant] (6.5,1.5) rectangle (8.5,2);
\fill[pimplant] (17,1.5) rectangle (18.75,2);
	\end{tikzpicture}
	\caption{Oxide removal}
\end{figure}
