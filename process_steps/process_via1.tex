\section{First vias (Contacts)}\label{via1}

This first set of vias connect the first metal layer to the active area.
These vias are in the fringe between front-end and back-end process.

\begin{figure}[H]
	\centering
	\begin{tikzpicture}[node distance = 3cm, auto, thick,scale=\CrossSectionOnly, every node/.style={transform shape}]
		\fill[isolationoxide] (0,1.5) rectangle (20,\LowerMetal1);

\input{tikz_process_steps/nimplant.a.tex}
\fill[pimplant] (3.0,1.5) rectangle (5,2);
\node at (4,1.65) {p+};
\fill[pimplant] (6.5,1.5) rectangle (8.5,2);
\node at (7,1.65) {p+};
\fill[pimplant] (17,1.5) rectangle (18.75,2);
\node at (18,1.65) {p+};

\filldraw[line width=0, isolationoxide] (5,2.0) -- (4.5,2.0) -- (5,3.0);
\filldraw[line width=0, isolationoxide] (6.5,2.0) -- (7.0,2.0) -- (6.5,3.0);
\filldraw[line width=0, isolationoxide] (13.5,2.0) -- (13.0,2.0) -- (13.5,3.0);
\filldraw[line width=0, isolationoxide] (15,2.0) -- (15.5,2.0) -- (15,3.0);

\fill[silicide] (1.5,1.8) rectangle (4.5,2);
\fill[silicide] (5,2.8) rectangle (6.5,3.0);
\fill[silicide] (7,1.8) rectangle (8,2);

\fill[silicide] (12,1.8) rectangle (13,2);
\fill[silicide] (13.5,2.8) rectangle (15,3.0);
\fill[silicide] (15.5,1.8) rectangle (18.5,2);

\fill[metal1] (1.5,2.0) rectangle (2.75,\LowerMetal1);
\fill[metal1] (3.25,2.0) rectangle (4.25,\LowerMetal1);
\fill[metal1] (5.25,3.0) rectangle (6.25,\LowerMetal1);
\fill[metal1] (7.25,2.0) rectangle (8.25,\LowerMetal1);
\fill[metal1] (12.0,2.0) rectangle (12.75,\LowerMetal1);
\fill[metal1] (13.75,3.0) rectangle (14.75,\LowerMetal1);
\fill[metal1] (15.75,2.0) rectangle (16.75,\LowerMetal1);
\fill[metal1] (17.25,2.0) rectangle (18.25,\LowerMetal1);

\fill[metal1] (0,\LowerMetal1) rectangle (20.0,\UpperMetal1);
	\end{tikzpicture}
	\caption{Contact geometry target}
	\label{via1_cross_sections}
\end{figure}

As can be seen in \autoref{via1_cross_sections} the goal of this step is purely to deposit a layer of isolation oxide, get the holes into it, down to the silicide and polyside in order to form wires later on.
After this step the whole surface will still be covered by Aluminum, that's why the top view has been omitted.for this step.

\begin{figure}[H]
	\centering
	\begin{tikzpicture}[node distance =1cm, auto, thick,scale=\VLSILayout, every node/.style={transform shape}]
		\input{tikz_process_steps/nimplant.layout.tex}

% p+
\fill[pimplant,opacity=\OpacityLayout] (3,0.75) rectangle (8.5,7.5);
\fill[pimplant,opacity=\OpacityLayout] (17,0.75) rectangle (19,7.5);
% gate metal
\fill[gatemetal,opacity=\OpacityLayout] (4.8,1.75) rectangle (6.7,8);
\fill[gatemetal,opacity=\OpacityLayout] (13.3,1.75) rectangle (15.2,8);
\fill[gatemetal,opacity=\OpacityLayout] (4.8,8) rectangle (15.2,10);

%vias
\fill[DarkOrchid,opacity=0.2] (7,3.75) rectangle (7.5,4.25);
\fill[DarkOrchid,opacity=0.2] (12.5,3.75) rectangle (13,4.25);
\fill[DarkOrchid,opacity=0.2] (9.75,8.75) rectangle (10.25,9.25);
\fill[DarkOrchid,opacity=0.2] (2.25,3.75) rectangle (2.75,4.25);
\fill[DarkOrchid,opacity=0.2] (3.75,3.75) rectangle (4.25,4.25);
\fill[DarkOrchid,opacity=0.2] (15.75,3.75) rectangle (16.25,4.25);
\fill[DarkOrchid,opacity=0.2] (17.25,3.75) rectangle (17.75,4.25);
	\end{tikzpicture}
	\caption{First via layout}
	\label{via1_layout}
\end{figure}

It should be noted again that the via placement and dimensions in \autoref{via1_layout} are solely for demonstration purposes for the process only and are in not way the actual standard cell design for the final standard cell lib. \\

In a later iteration of this process we might be switching to Tungsten as the metal for this step.

\newpage

\subsection{Isolation dioxide layer}

We now need to grow a layer of thick oxide in order to isolate the Aluminum interconnect layer from the active area.

\begin{figure}[H]
	\centering
	\begin{tikzpicture}[node distance = 3cm, auto, thick,scale=\CrossSectionOnly, every node/.style={transform shape}]
		\input{tikz_process_steps/nimplant.a.tex}
\fill[pimplant] (3.0,1.5) rectangle (5,2);
\node at (4,1.65) {p+};
\fill[pimplant] (6.5,1.5) rectangle (8.5,2);
\node at (7,1.65) {p+};
\fill[pimplant] (17,1.5) rectangle (18.75,2);
\node at (18,1.65) {p+};

\filldraw[line width=0, isolationoxide] (5,2.0) -- (4.5,2.0) -- (5,3.0);
\filldraw[line width=0, isolationoxide] (6.5,2.0) -- (7.0,2.0) -- (6.5,3.0);
\filldraw[line width=0, isolationoxide] (13.5,2.0) -- (13.0,2.0) -- (13.5,3.0);
\filldraw[line width=0, isolationoxide] (15,2.0) -- (15.5,2.0) -- (15,3.0);

\fill[silicide] (1.5,1.8) rectangle (4.5,2);
\fill[silicide] (5,2.8) rectangle (6.5,3.0);
\fill[silicide] (7,1.8) rectangle (8,2);

\fill[silicide] (12,1.8) rectangle (13,2);
\fill[silicide] (13.5,2.8) rectangle (15,3.0);
\fill[silicide] (15.5,1.8) rectangle (18.5,2);
	\end{tikzpicture}
	\drawStepArrow{LTO deposition}
	\begin{tikzpicture}[node distance = 3cm, auto, thick,scale=\CrossSectionOnly, every node/.style={transform shape}]
		\fill[isolationoxide] (0,0) rectangle (20,\LowerMetal1);
\input{tikz_process_steps/nimplant.a.tex}
\fill[pimplant] (3.0,1.5) rectangle (5,2);
\node at (4,1.65) {p+};
\fill[pimplant] (6.5,1.5) rectangle (8.5,2);
\node at (7,1.65) {p+};
\fill[pimplant] (17,1.5) rectangle (18.75,2);
\node at (18,1.65) {p+};

\filldraw[line width=0, isolationoxide] (5,2.0) -- (4.5,2.0) -- (5,3.0);
\filldraw[line width=0, isolationoxide] (6.5,2.0) -- (7.0,2.0) -- (6.5,3.0);
\filldraw[line width=0, isolationoxide] (13.5,2.0) -- (13.0,2.0) -- (13.5,3.0);
\filldraw[line width=0, isolationoxide] (15,2.0) -- (15.5,2.0) -- (15,3.0);

\fill[silicide] (1.5,1.8) rectangle (4.5,2);
\fill[silicide] (5,2.8) rectangle (6.5,3.0);
\fill[silicide] (7,1.8) rectangle (8,2);

\fill[silicide] (12,1.8) rectangle (13,2);
\fill[silicide] (13.5,2.8) rectangle (15,3.0);
\fill[silicide] (15.5,1.8) rectangle (18.5,2);
	\end{tikzpicture}
	\caption{Oxide layer}
\end{figure}

\subsection{Pattering}

The resist is being deposited using spin coating and then baked depending on the baking time for the specific resist.
The layout for being exposed onto the resist is being extracted from the "n\_plus\_select" layer within the GDS2 file onto a \textbf{bright field} mask.
The requirement is a \textbf{negative} tone resist.

\begin{figure}[H]
	\centering
	\begin{tikzpicture}[node distance = 3cm, auto, thick,scale=\CrossAndTopSection, every node/.style={transform shape}]
		\input{tikz_process_steps/via1.patterning.a.tex}
	\end{tikzpicture}
	\begin{tikzpicture}[node distance = 3cm, auto, thick,scale=\CrossAndTopSection, every node/.style={transform shape}]
		\input{tikz_process_steps/via1.patterning.at.tex}
	\end{tikzpicture}
	\drawStepArrow{Mask: nselect}
	\begin{tikzpicture}[node distance = 3cm, auto, thick,scale=\CrossAndTopSection, every node/.style={transform shape}]
		\input{tikz_process_steps/via1.patterning.b.tex}
	\end{tikzpicture}
	\begin{tikzpicture}[node distance = 3cm, auto, thick,scale=\CrossAndTopSection, every node/.style={transform shape}]
		\input{tikz_process_steps/via1.patterning.bt.tex}
	\end{tikzpicture}
	\caption{N+ region resist mask}
\end{figure}

The thickness of the resist layer and the baking duration will variate depending on the specific equipment for which this process will be implemented with.
Also after the exposure and development, the hard baking shouldn't be forgotten!

\subsection{Etching}

We now need to open a window in the dioxide layer, through which we will inject carrier atoms into the silicon crystal structure.

\begin{figure}[H]
	\centering
	\begin{tikzpicture}[node distance = 3cm, auto, thick,scale=\CrossAndTopSection, every node/.style={transform shape}]
		\input{tikz_process_steps/via1.etching.a.tex}
	\end{tikzpicture}
	\begin{tikzpicture}[node distance = 3cm, auto, thick,scale=\CrossAndTopSection, every node/.style={transform shape}]
		\input{tikz_process_steps/via1.etching.at.tex}
	\end{tikzpicture}
	\drawStepArrow{}
	\begin{tikzpicture}[node distance = 3cm, auto, thick,scale=\CrossAndTopSection, every node/.style={transform shape}]
		\input{tikz_process_steps/via1.etching.b.tex}
	\end{tikzpicture}
	\begin{tikzpicture}[node distance = 3cm, auto, thick,scale=\CrossAndTopSection, every node/.style={transform shape}]
		\input{tikz_process_steps/via1.etching.bt.tex}
	\end{tikzpicture}
	\caption{N+ region opened}
\end{figure}

Since the silicon dioxide layer is 100nm thick and we wanna reach the silicon below we can use wet etching as described in the chemistry chapter.\\

\textbf{Possible approaches}:
\begin{itemize}
	\item \textbf{"AOE Etcher (DRY-AOE)" from HKUST} \\
	We can use anisotropic plasma etching for sharper borders.
	\item \textbf{Chemical solution} \\
	We can use buffered hydrofluoric acid (BOE (1:6)) at room temperature for a little bit over 3 minutes in order to get through the 100nm of oxide.\\
	Too long over 1 minutes might cause under-etch however!
\end{itemize}

\subsection{Cleaning}
\begin{figure}[H]
	\centering
	\begin{tikzpicture}[node distance = 3cm, auto, thick,scale=\CrossSectionOnly, every node/.style={transform shape}]
		\input{tikz_process_steps/via1.cleaning.a.tex}
	\end{tikzpicture}
	\drawStepArrow{}
	\begin{tikzpicture}[node distance = 3cm, auto, thick,scale=\CrossSectionOnly, every node/.style={transform shape}]
		\input{tikz_process_steps/via1.cleaning.b.tex}
	\end{tikzpicture}
	\caption{Resist removal}
\end{figure}