\section{N-well}\label{nwell_chapter}
In order to build CMOS on the same substrate, an N-well is required for building the complementary P-channel transistor for a n-p-channel logic circuitry as shown above in the example section.
The cross section as well as the top view of the targeted geometry are shown in \autoref{nwell_target}
\begin{figure}[H]
	\centering
	\begin{tikzpicture}[node distance = 3cm, auto, thick,scale=\CrossAndTopSectionBig, every node/.style={transform shape}]
		% substrate
\fill[substrate] (0,0) rectangle (20,2);
\node at (2,0.5) {Silicon substrate};
%trenches
\fill[isolationoxide] (0,0.75) rectangle (1,2);
\fill[isolationoxide] (8.5,0.75) rectangle (11.5,2);
\fill[isolationoxide] (19,0.75) rectangle (20,2);
% n-well
\fill[nwell] (1.25,0.75) rectangle (8.5,2);
\node at (5.75,1) {N-Well};


	\end{tikzpicture}
	\begin{tikzpicture}[node distance = 3cm, auto, thick,scale=\CrossAndTopSectionBig, every node/.style={transform shape}]
		% substrate
\fill[YellowOrange] (0,0) rectangle (20,12);
% trench area
\fill[DarkGray] (0,0) rectangle (1,12);
\fill[DarkGray] (8.5,0) rectangle (11.5,12);
\fill[DarkGray] (19,0) rectangle (20,12);
\fill[DarkGray] (0,0) rectangle (20,1.25);
\fill[DarkGray] (0,7.5) rectangle (20,12);
\fill[nwell] (1.25,1) rectangle (8.25,7.25);
	\end{tikzpicture}
	\caption{N-well target geometry}
	\label{nwell_target}
\end{figure}

The N-well will serve us as an island of N-doped substrate within the P-doped basis substrate.

The dopant dose will be $2.5\times10^{12}cm^{-2}$ as calculated in the documentation of the process design leading to these steps\footnote{\url{https://github.com/leviathanch/libresiliconprocess/raw/master/process_design/process_design.pdf}}.

\begin{figure}[H]
	\centering
	\begin{tikzpicture}[node distance =1cm, auto, thick,scale=\VLSILayout, every node/.style={transform shape}]
		\fill[nwell,opacity=0.2] (0.75,0.5) rectangle (8.75,7.75);
\fill[nwell,opacity=0.2] (11.25,0.5) rectangle (19.25,7.75);

\draw[dotted] (20.5,0.5) rectangle (25,6.5);

\node at (22.25,6) {\textbf{Layers}};

\fill[nwell,opacity=0.2] (21,1) rectangle (21.5,1.5);
\node at (22.25,1.25) {active};

\fill[resist,opacity=0.2] (21,1.5) rectangle (21.5,2);
\node at (22.25,1.75) {nwell};

\fill[blue,opacity=0.2] (21,2) rectangle (21.5,2.5);
\node at (22.25,2.25) {nimplant};

\fill[nitride,opacity=0.2] (21,2.5) rectangle (21.5,3);
\node at (22.25,2.75) {pimplant};

\fill[Emerald,opacity=0.2] (21,3) rectangle (21.5,3.5);
\node at (22.25,3.25) {gate};

\fill[Fuchsia,opacity=0.2] (21,3.5) rectangle (21.5,4);
\node at (22.25,3.75) {metal1};

\fill[DarkOrchid,opacity=0.2] (21,4) rectangle (21.5,4.5);
\node at (22.25,4.25) {via1};

\fill[LimeGreen,opacity=0.2] (21,4.5) rectangle (21.5,5);
\node at (22.25,4.75) {metal2};

\fill[ForestGreen,opacity=0.2] (21,5) rectangle (21.5,5.5);
\node at (22.25,5.25) {via2};

\fill[nwell,opacity=\OpacityLayout] (1,0.75) rectangle (8.5,7.5);
	\end{tikzpicture}
	\caption{N-Well layout}
	\label{nwell_layout}
\end{figure}

In \autoref{nwell_layout} the layout of the n-well region on top of the active area region can be seen.

The n-well is being fit into the active area. It should even be a little bit bigger than the active area, because of possible alignment offsets

\newpage

\subsection{Hard mask: Dioxide growth}

In order to selectively inject charge carrying atoms into the crystalline structure a protective dioxide ($SiO_2$) layer needs to be grown on top of a p-type substrate.

\begin{figure}[H]
	\centering
	\begin{tikzpicture}[node distance = 3cm, auto, thick,scale=\CrossSectionOnly, every node/.style={transform shape}]
		\input{tikz_process_steps/nwell.a.tex}

% pwell
\fill[pwell] (11.75,0.75) rectangle (18.75,2);
\node at (14.25,1) {P-Well};
	\end{tikzpicture}
	\drawStepArrow{Wet oxidation}
	\begin{tikzpicture}[node distance = 3cm, auto, thick,scale=\CrossSectionOnly, every node/.style={transform shape}]
		% oxide
\fill[isolationoxide] (0,1.25) rectangle (20,2.75);

% oxide hill 1
\fill[isolationoxide] (1.25,2.75) rectangle (8.25,3.5);
\filldraw[line width=0, isolationoxide] (0.5,2.75) -- (1.25,2.75) -- (1.25,3.5);
\filldraw[line width=0, isolationoxide] (8.25,2.75) -- (8.25,3.5) -- (9.0,2.75);

% oxide hill 2
\fill[isolationoxide] (11.75,2.75) rectangle (18.75,3.5);
\filldraw[line width=0, isolationoxide] (11.0,2.75) -- (11.75,2.75) -- (11.75,3.5);
\filldraw[line width=0, isolationoxide] (18.75,2.75) -- (18.75,3.5) -- (19.5,2.75);

\node at (2,2.1) {SiO2};

% substrate
\fill[substrate] (0,0) rectangle (20,2);
\node at (2,0.5) {Silicon substrate};
%trenches
\fill[isolationoxide] (0,0.75) rectangle (1,2);
\fill[isolationoxide] (8.5,0.75) rectangle (11.5,2);
\fill[isolationoxide] (19,0.75) rectangle (20,2);
	\end{tikzpicture}
	\caption{Dioxide layer growth}
\end{figure}

With an energy of 100keV for the implantation performed in \autoref{nwell_implant_step}, the projected range of the dopants within the oxide will be 100nm (130nm tops) \footnote{\url{http://cleanroom.byu.edu/rangestraggle}}.
This means being on the safe side and having 200nm as the thickness is a good approach.

In order to grow the 200nm thick oxide layer, the wafer is being oxidized for around 14 minutes at 1050\degree C using wet oxidation which results in a dioxide layer of around 200nm in thickness\footnote{\url{http://cleanroom.byu.edu/OxideTimeCalc}}.

\subsection{Hard mask: Patterning}

The resist is being deposited using spray coating because the uneven nature of the oxide layer.
After that the wafer is being soft baked depending on the baking time and temperature for the specific resist.
The layout for being exposed onto the resist is being extracted from the "nwell" layer within the GDS2 file onto a \textbf{bright field} mask.
The requirement is a \textbf{negative} tone resist.

\begin{figure}[H]
	\centering
	\begin{tikzpicture}[node distance = 3cm, auto, thick,scale=\CrossAndTopSection, every node/.style={transform shape}]
		% oxide
\fill[isolationoxide] (0,1.25) rectangle (20,2.75);

% oxide hill 1
\fill[isolationoxide] (1.25,2.75) rectangle (8.25,3.5);
\filldraw[line width=0, isolationoxide] (0.5,2.75) -- (1.25,2.75) -- (1.25,3.5);
\filldraw[line width=0, isolationoxide] (8.25,2.75) -- (8.25,3.5) -- (9.0,2.75);

% oxide hill 2
\fill[isolationoxide] (11.75,2.75) rectangle (18.75,3.5);
\filldraw[line width=0, isolationoxide] (11.0,2.75) -- (11.75,2.75) -- (11.75,3.5);
\filldraw[line width=0, isolationoxide] (18.75,2.75) -- (18.75,3.5) -- (19.5,2.75);

\node at (2,2.1) {SiO2};

% substrate
\fill[substrate] (0,0) rectangle (20,2);
\node at (2,0.5) {Silicon substrate};
%trenches
\fill[isolationoxide] (0,0.75) rectangle (1,2);
\fill[isolationoxide] (8.5,0.75) rectangle (11.5,2);
\fill[isolationoxide] (19,0.75) rectangle (20,2);
	\end{tikzpicture}
	\begin{tikzpicture}[node distance = 3cm, auto, thick,scale=\CrossAndTopSection, every node/.style={transform shape}]
		% resist
\fill[isolationoxide] (0,0) rectangle (20,12);
	\end{tikzpicture}
	\drawStepArrow{Mask: nwell}
	\begin{tikzpicture}[node distance = 3cm, auto, thick,scale=\CrossAndTopSection, every node/.style={transform shape}]
		% resist
\fill[resist] (0,2.6) rectangle (1,5.0);
\fill[resist] (8.5,2.6) rectangle (20,5.0);

% oxide
\fill[isolationoxide] (0,1.25) rectangle (20,2.75);

% oxide hill 1
\fill[isolationoxide] (1.25,2.75) rectangle (8.25,3.5);
\filldraw[line width=0, isolationoxide] (0.5,2.75) -- (1.25,2.75) -- (1.25,3.5);
\filldraw[line width=0, isolationoxide] (8.25,2.75) -- (8.25,3.5) -- (9.0,2.75);

% oxide hill 2
\fill[isolationoxide] (11.75,2.75) rectangle (18.75,3.5);
\filldraw[line width=0, isolationoxide] (11.0,2.75) -- (11.75,2.75) -- (11.75,3.5);
\filldraw[line width=0, isolationoxide] (18.75,2.75) -- (18.75,3.5) -- (19.5,2.75);

\node at (2,2.1) {SiO2};

% substrate
\fill[substrate] (0,0) rectangle (20,2);
\node at (2,0.5) {Silicon substrate};
%trenches
\fill[isolationoxide] (0,0.75) rectangle (1,2);
\fill[isolationoxide] (8.5,0.75) rectangle (11.5,2);
\fill[isolationoxide] (19,0.75) rectangle (20,2);
	\end{tikzpicture}
	\begin{tikzpicture}[node distance = 3cm, auto, thick,scale=\CrossAndTopSection, every node/.style={transform shape}]
		% resist
\fill[resist] (0,0) rectangle (20,12);
% substrate
\fill[isolationoxide] (1.25,1.5) rectangle (8.25,7.25);
	\end{tikzpicture}
	\caption{Cross/top view of n-well layout on resist}
\end{figure}

The thickness of the resist layer and the baking duration will variate depending on the specific equipment for which this process will be implemented with.
Also after the exposure and development, the hard baking shouldn't be forgotten!

\newpage

\subsection{Hard mask: Etching}

We now need to open a window in the dioxide layer, through which we will inject carrier atoms into the silicon crystal structure.

\begin{figure}[H]
	\centering
	\begin{tikzpicture}[node distance = 3cm, auto, thick,scale=\CrossAndTopSection, every node/.style={transform shape}]
		% resist
\fill[resist] (0,2.6) rectangle (1,5.0);
\fill[resist] (8.5,2.6) rectangle (20,5.0);

% oxide
\fill[isolationoxide] (0,1.25) rectangle (20,2.75);

% oxide hill 1
\fill[isolationoxide] (1.25,2.75) rectangle (8.25,3.5);
\filldraw[line width=0, isolationoxide] (0.5,2.75) -- (1.25,2.75) -- (1.25,3.5);
\filldraw[line width=0, isolationoxide] (8.25,2.75) -- (8.25,3.5) -- (9.0,2.75);

% oxide hill 2
\fill[isolationoxide] (11.75,2.75) rectangle (18.75,3.5);
\filldraw[line width=0, isolationoxide] (11.0,2.75) -- (11.75,2.75) -- (11.75,3.5);
\filldraw[line width=0, isolationoxide] (18.75,2.75) -- (18.75,3.5) -- (19.5,2.75);

\node at (2,2.1) {SiO2};

\input{tikz_process_steps/sti.a.tex}
	\end{tikzpicture}
	\begin{tikzpicture}[node distance = 3cm, auto, thick,scale=\CrossAndTopSection, every node/.style={transform shape}]
		% resist
\fill[resist] (0,0) rectangle (20,12);
% substrate
\fill[isolationoxide]  (1,0.75) rectangle (8.5,7.5);
	\end{tikzpicture}
	\drawStepArrow{}
	\begin{tikzpicture}[node distance = 3cm, auto, thick,scale=\CrossAndTopSection, every node/.style={transform shape}]
		% resist
\fill[resist] (0,2.6) rectangle (1,5.0);
\fill[resist] (8.5,2.6) rectangle (20,5.0);

% oxide
\fill[isolationoxide] (0,1.25) rectangle (1,2.75);
\fill[isolationoxide] (8.5,1.25) rectangle (20,2.75);

% oxide hill 1
\filldraw[line width=0, isolationoxide] (0.5,2.75) -- (1.0,2.75) -- (1.0,3.0);
\filldraw[line width=0, isolationoxide] (8.75,2.75) -- (8.5,2.75) -- (8.5,3.0);

% oxide hill 2
\fill[isolationoxide] (11.75,2.75) rectangle (18.75,3.5);
\filldraw[line width=0, isolationoxide] (11.0,2.75) -- (11.75,2.75) -- (11.75,3.5);
\filldraw[line width=0, isolationoxide] (18.75,2.75) -- (18.75,3.5) -- (19.5,2.75);

% substrate
\fill[substrate] (0,0) rectangle (20,2);
\node at (2,0.5) {Silicon substrate};
%trenches
\fill[isolationoxide] (0,0.75) rectangle (1,2);
\fill[isolationoxide] (8.5,0.75) rectangle (11.5,2);
\fill[isolationoxide] (19,0.75) rectangle (20,2);
	\end{tikzpicture}
	\begin{tikzpicture}[node distance = 3cm, auto, thick,scale=\CrossAndTopSection, every node/.style={transform shape}]
		% resist
\fill[resist] (0,0) rectangle (20,12);
% substrate
\fill[substrate] (1.25,1) rectangle (8.25,7.25);
	\end{tikzpicture}
	\caption{Cross/top view of n-well oxide window}
\end{figure}

Since the silicon dioxide layer is 300nm thick and we wanna reach the silicon below we can use wet etching as described in the chemistry chapter.\\

\textbf{Possible approaches}:
\begin{itemize}
	\item \textbf{"AOE Etcher (DRY-AOE)" from HKUST} \\
	We can use anisotropic plasma etching for sharper borders.
	\item \textbf{Chemical solution} \\
	We can use buffered hydrofluoric acid (BOE (1:6)) at room temperature for a little bit over 3 minutes in order to get through the 300nm of oxide.\\
	Too long over 3 minutes might cause under-etch however!
\end{itemize}

\subsection{Hard mask: Resist strip}
In order to avoid contamination of the machines we need to make sure all the resist has been stripped off from the wafer.
\begin{figure}[H]
	\centering
	\begin{tikzpicture}[node distance = 3cm, auto, thick,scale=\CrossSectionOnly, every node/.style={transform shape}]
		% resist
\fill[resist] (0,2.6) rectangle (1,5.0);
\fill[resist] (8.5,2.6) rectangle (20,5.0);

% oxide
\fill[isolationoxide] (0,1.25) rectangle (1,2.75);
\fill[isolationoxide] (8.5,1.25) rectangle (20,2.75);

% oxide hill 1
\filldraw[line width=0, isolationoxide] (0.5,2.75) -- (1.0,2.75) -- (1.0,3.0);
\filldraw[line width=0, isolationoxide] (8.75,2.75) -- (8.5,2.75) -- (8.5,3.0);

% oxide hill 2
\fill[isolationoxide] (11.75,2.75) rectangle (18.75,3.5);
\filldraw[line width=0, isolationoxide] (11.0,2.75) -- (11.75,2.75) -- (11.75,3.5);
\filldraw[line width=0, isolationoxide] (18.75,2.75) -- (18.75,3.5) -- (19.5,2.75);

% substrate
\fill[substrate] (0,0) rectangle (20,2);
\node at (2,0.5) {Silicon substrate};
%trenches
\fill[isolationoxide] (0,0.75) rectangle (1,2);
\fill[isolationoxide] (8.5,0.75) rectangle (11.5,2);
\fill[isolationoxide] (19,0.75) rectangle (20,2);
	\end{tikzpicture}
	\drawStepArrow{}
	\begin{tikzpicture}[node distance = 3cm, auto, thick,scale=\CrossSectionOnly, every node/.style={transform shape}]
		% oxide
\fill[isolationoxide] (0,1.25) rectangle (1,2.75);
\fill[isolationoxide] (8.5,1.25) rectangle (20,2.75);

% oxide hill 1
\filldraw[line width=0, isolationoxide] (0.5,2.75) -- (1.0,2.75) -- (1.0,3.0);
\filldraw[line width=0, isolationoxide] (8.75,2.75) -- (8.5,2.75) -- (8.5,3.0);

% oxide hill 2
\fill[isolationoxide] (11.75,2.75) rectangle (18.75,3.5);
\filldraw[line width=0, isolationoxide] (11.0,2.75) -- (11.75,2.75) -- (11.75,3.5);
\filldraw[line width=0, isolationoxide] (18.75,2.75) -- (18.75,3.5) -- (19.5,2.75);

% substrate
\fill[substrate] (0,0) rectangle (20,2);
\node at (2,0.5) {Silicon substrate};
%trenches
\fill[isolationoxide] (0,0.75) rectangle (1,2);
\fill[isolationoxide] (8.5,0.75) rectangle (11.5,2);
\fill[isolationoxide] (19,0.75) rectangle (20,2);
	\end{tikzpicture}
	\caption{Resist removal}
\end{figure}
Please just use the solvent for the specific resist.

\newpage

\subsection{Implantation/Doping}\label{nwell_implant_step}
We now need to inject the carriers into the upper level of the n-channel area so that we can later on drive them into the crystal during the drive-in step.

\begin{figure}[H]
	\centering
	\begin{minipage}{0.5\textwidth}
	\centering
	\begin{tikzpicture}[node distance = 3cm, auto, thick,scale=\CrossSectionOnly, every node/.style={transform shape}]
		% oxide
\fill[isolationoxide] (0,1.25) rectangle (1,2.75);
\fill[isolationoxide] (8.5,1.25) rectangle (20,2.75);

% oxide hill 1
\filldraw[line width=0, isolationoxide] (0.5,2.75) -- (1.0,2.75) -- (1.0,3.0);
\filldraw[line width=0, isolationoxide] (8.75,2.75) -- (8.5,2.75) -- (8.5,3.0);

% oxide hill 2
\fill[isolationoxide] (11.75,2.75) rectangle (18.75,3.5);
\filldraw[line width=0, isolationoxide] (11.0,2.75) -- (11.75,2.75) -- (11.75,3.5);
\filldraw[line width=0, isolationoxide] (18.75,2.75) -- (18.75,3.5) -- (19.5,2.75);

% substrate
\fill[substrate] (0,0) rectangle (20,2);
\node at (2,0.5) {Silicon substrate};
%trenches
\fill[isolationoxide] (0,0.75) rectangle (1,2);
\fill[isolationoxide] (8.5,0.75) rectangle (11.5,2);
\fill[isolationoxide] (19,0.75) rectangle (20,2);

\forloop{ct}{0}{\value{ct} < 21}
{
	\draw [->] (\value{ct},5) -- (\value{ct},4);
	\node at (\value{ct},5.2) {P$^{31}$};
}
	\end{tikzpicture}
	\drawStepArrow{Boron implant}
	\begin{tikzpicture}[node distance = 3cm, auto, thick,scale=\CrossSectionOnly, every node/.style={transform shape}]
		% resist
\fill[resist] (0.25,2.0) rectangle (1,5.0);
\fill[resist] (8.5,2.0) rectangle (19.75,5.0);

% oxide
\fill[isolationoxide] (0,1.25) rectangle (1,2.75);
\fill[isolationoxide] (8.5,1.25) rectangle (20,2.75);

% oxide hill 1
\filldraw[line width=0, isolationoxide] (0.5,2.75) -- (1.0,2.75) -- (1.0,3.0);
\filldraw[line width=0, isolationoxide] (8.75,2.75) -- (8.5,2.75) -- (8.5,3.0);

% oxide hill 2
\fill[isolationoxide] (11.75,2.75) rectangle (18.75,3.5);
\filldraw[line width=0, isolationoxide] (11.0,2.75) -- (11.75,2.75) -- (11.75,3.5);
\filldraw[line width=0, isolationoxide] (18.75,2.75) -- (18.75,3.5) -- (19.5,2.75);

% substrate
\fill[substrate] (0,0) rectangle (20,2);
\node at (2,0.5) {Silicon substrate};
%trenches
\fill[isolationoxide] (0,0.75) rectangle (1,2);
\fill[isolationoxide] (8.5,0.75) rectangle (11.5,2);
\fill[isolationoxide] (19,0.75) rectangle (20,2);

	\end{tikzpicture}
	\drawStepArrow{Annealing}
	\begin{tikzpicture}[node distance = 3cm, auto, thick,scale=\CrossSectionOnly, every node/.style={transform shape}]
		% oxide
\fill[isolationoxide] (0,1.25) rectangle (1,2.75);
\fill[isolationoxide] (8.5,1.25) rectangle (20,2.75);

% oxide hill 1
\filldraw[line width=0, isolationoxide] (0.5,2.75) -- (1.0,2.75) -- (1.0,3.0);
\filldraw[line width=0, isolationoxide] (8.75,2.75) -- (8.5,2.75) -- (8.5,3.0);

% oxide hill 2
\fill[isolationoxide] (11.75,2.75) rectangle (18.75,3.5);
\filldraw[line width=0, isolationoxide] (11.0,2.75) -- (11.75,2.75) -- (11.75,3.5);
\filldraw[line width=0, isolationoxide] (18.75,2.75) -- (18.75,3.5) -- (19.5,2.75);

% substrate
\fill[substrate] (0,0) rectangle (20,2);
\node at (2,0.5) {Silicon substrate};
%trenches
\fill[isolationoxide] (0,0.75) rectangle (1,2);
\fill[isolationoxide] (8.5,0.75) rectangle (11.5,2);
\fill[isolationoxide] (19,0.75) rectangle (20,2);
\shade[upper left = nwell, upper right = nwell, lower right = substrate, lower left = substrate,] (1.25,1.5) rectangle (8.25,2.0);
\shade[upper left = pwell, upper right = pwell, lower right = substrate, lower left = substrate,] (11.75,1.0) rectangle (18.75,2.0);
	\end{tikzpicture} \\
	\textbf{Implantation approach}
	\end{minipage}\begin{minipage}{0.5\textwidth}
	\centering
	\begin{tikzpicture}[node distance = 3cm, auto, thick,scale=\CrossSectionOnly, every node/.style={transform shape}]
		\fill[nwell] (0,0) rectangle (20,5);

% oxide
\fill[isolationoxide] (0,1.25) rectangle (1,2.75);
\fill[isolationoxide] (8.5,1.25) rectangle (20,2.75);

% oxide hill 1
\filldraw[line width=0, isolationoxide] (0.5,2.75) -- (1.0,2.75) -- (1.0,3.0);
\filldraw[line width=0, isolationoxide] (8.75,2.75) -- (8.5,2.75) -- (8.5,3.0);

% oxide hill 2
\fill[isolationoxide] (11.75,2.75) rectangle (18.75,3.5);
\filldraw[line width=0, isolationoxide] (11.0,2.75) -- (11.75,2.75) -- (11.75,3.5);
\filldraw[line width=0, isolationoxide] (18.75,2.75) -- (18.75,3.5) -- (19.5,2.75);

% substrate
\fill[substrate] (0,0) rectangle (20,2);
\node at (2,0.5) {Silicon substrate};
%trenches
\fill[isolationoxide] (0,0.75) rectangle (1,2);
\fill[isolationoxide] (8.5,0.75) rectangle (11.5,2);
\fill[isolationoxide] (19,0.75) rectangle (20,2);
	\end{tikzpicture}
	\drawStepArrow{Constant source diffusion}
	\begin{tikzpicture}[node distance = 3cm, auto, thick,scale=\CrossSectionOnly, every node/.style={transform shape}]
		\fill[nwell] (0,0) rectangle (20,5);

% oxide
\fill[isolationoxide] (0,1.25) rectangle (1,2.75);
\fill[isolationoxide] (8.5,1.25) rectangle (20,2.75);

% oxide hill 1
\filldraw[line width=0, isolationoxide] (0.5,2.75) -- (1.0,2.75) -- (1.0,3.0);
\filldraw[line width=0, isolationoxide] (8.75,2.75) -- (8.5,2.75) -- (8.5,3.0);

% oxide hill 2
\fill[isolationoxide] (11.75,2.75) rectangle (18.75,3.5);
\filldraw[line width=0, isolationoxide] (11.0,2.75) -- (11.75,2.75) -- (11.75,3.5);
\filldraw[line width=0, isolationoxide] (18.75,2.75) -- (18.75,3.5) -- (19.5,2.75);

% substrate
\fill[substrate] (0,0) rectangle (20,2);
\node at (2,0.5) {Silicon substrate};
%trenches
\fill[isolationoxide] (0,0.75) rectangle (1,2);
\fill[isolationoxide] (8.5,0.75) rectangle (11.5,2);
\fill[isolationoxide] (19,0.75) rectangle (20,2);

\shade[upper left = nwell, upper right = nwell, lower right = substrate, lower left = substrate,] (1.25,1.5) rectangle (8.25,2.0);
\shade[upper left = pwell, upper right = pwell, lower right = substrate, lower left = substrate,] (11.75,1.0) rectangle (18.75,2.0);
	\end{tikzpicture}
	\drawStepArrow{Source removal}
	\begin{tikzpicture}[node distance = 3cm, auto, thick,scale=\CrossSectionOnly, every node/.style={transform shape}]
		% oxide
\fill[isolationoxide] (0,1.25) rectangle (1,2.75);
\fill[isolationoxide] (8.5,1.25) rectangle (20,2.75);

% oxide hill 1
\filldraw[line width=0, isolationoxide] (0.5,2.75) -- (1.0,2.75) -- (1.0,3.0);
\filldraw[line width=0, isolationoxide] (8.75,2.75) -- (8.5,2.75) -- (8.5,3.0);

% oxide hill 2
\fill[isolationoxide] (11.75,2.75) rectangle (18.75,3.5);
\filldraw[line width=0, isolationoxide] (11.0,2.75) -- (11.75,2.75) -- (11.75,3.5);
\filldraw[line width=0, isolationoxide] (18.75,2.75) -- (18.75,3.5) -- (19.5,2.75);

% substrate
\fill[substrate] (0,0) rectangle (20,2);
\node at (2,0.5) {Silicon substrate};
%trenches
\fill[isolationoxide] (0,0.75) rectangle (1,2);
\fill[isolationoxide] (8.5,0.75) rectangle (11.5,2);
\fill[isolationoxide] (19,0.75) rectangle (20,2);

\shade[upper left = nwell, upper right = nwell, lower right = substrate, lower left = substrate,] (1.25,1.5) rectangle (8.25,2.0);
\shade[upper left = pwell, upper right = pwell, lower right = substrate, lower left = substrate,] (11.75,1.0) rectangle (18.75,2.0);
	\end{tikzpicture} \\
	\textbf{Diffusion approach}
	\end{minipage}
	\caption{Doping process}
\end{figure}

\textbf{Possible approaches}:
\begin{itemize}
	\item \textbf{"CF-3000 Implanter (IMP-3000)" from HKUST} \\
	At HKUST we have an implanter which gives us better control over the initial surface concentration. \\
	These steps are needed to arrive with the desired geometry:
	\begin{enumerate}
		\item The N-well is implanted with a Phosphorus ($P^{31}$) dose of $2.5\times10^{12}cm^{-2}$ at an energy of 100 keV.
		\item The N-well is annealed for 30 minutes at 1050\degreesC in $N_2$ environment (DIF-A1)\\
		After that the P-well will be around 2\um deep and the N-Well around 1\um deep
	\end{enumerate}
	\item \textbf{Constant source diffusion} \\
	We can add a layer of Phosphorus solution and diffusing in order to have an initial concentration in order to reach the desired concentration later by main diffusion.
		\begin{enumerate}
		\item A constant source is added (gas or liquid)
		\item The source dopant is driven in for 10 minutes at 1050\degreesC
		\item The dopant source is removed by stopping the gas flow or cleaning the surface
	\end{enumerate}
\end{itemize}

\subsection{Hard mask: Oxide removal}

Now we want to remove the silicon mask from the wafer and clean it for another clean oxide mask layer.

\begin{figure}[H]
	\centering
	\begin{tikzpicture}[node distance = 3cm, auto, thick,scale=\CrossSectionOnly, every node/.style={transform shape}]
		\fill[isolationoxide] (0,0) rectangle (20,2.25);
% resist
\fill[resist] (0.25,2.0) rectangle (1,5.0);
\fill[resist] (8.5,2.0) rectangle (19.75,5.0);

% oxide
\fill[isolationoxide] (0,1.25) rectangle (1,2.75);
\fill[isolationoxide] (8.5,1.25) rectangle (20,2.75);

% oxide hill 1
\filldraw[line width=0, isolationoxide] (0.5,2.75) -- (1.0,2.75) -- (1.0,3.0);
\filldraw[line width=0, isolationoxide] (8.75,2.75) -- (8.5,2.75) -- (8.5,3.0);

% oxide hill 2
\fill[isolationoxide] (11.75,2.75) rectangle (18.75,3.5);
\filldraw[line width=0, isolationoxide] (11.0,2.75) -- (11.75,2.75) -- (11.75,3.5);
\filldraw[line width=0, isolationoxide] (18.75,2.75) -- (18.75,3.5) -- (19.5,2.75);

\input{tikz_process_steps/sti.a.tex}


% n-well
\fill[nwell] (1.25,0.75) rectangle (8.25,2);
\node at (4.75,1) {N-Well};

\fill[pwell] (11.75,0.75) rectangle (18.75,2);
\node at (15.25,1) {P-Well};
	\end{tikzpicture}
	\drawStepArrow{}
	\begin{tikzpicture}[node distance = 3cm, auto, thick,scale=\CrossSectionOnly, every node/.style={transform shape}]
		% substrate
\fill[substrate] (0,0) rectangle (20,2);
\node at (2,0.5) {Silicon substrate};
%trenches
\fill[isolationoxide] (0,0.75) rectangle (1,2);
\fill[isolationoxide] (8.5,0.75) rectangle (11.5,2);
\fill[isolationoxide] (19,0.75) rectangle (20,2);

% n-well
\fill[nwell] (1.25,0.75) rectangle (8.25,2);
\node at (4.75,1) {N-Well};
	\end{tikzpicture}
	\caption{Oxide removal}
\end{figure}

We use buffered hydrofluoric acid (BOE (1:6)) at room temperature for 3 minutes in order to remove the 300nm of oxide layer.
