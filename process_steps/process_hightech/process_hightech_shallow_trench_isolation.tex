\section{Shallow trench isolation}\label{sti_chapter}
The geometry of a substrate with STI implemented can be seen in \autoref{sti_target}.

\begin{figure}[H]
	\centering
	\begin{tikzpicture}[node distance = 3cm, auto, thick,scale=\CrossAndTopSectionBig, every node/.style={transform shape}]
		% substrate
\fill[substrate] (0,0) rectangle (20,2);
\node at (2,0.5) {Silicon substrate};
%trenches
\fill[isolationoxide] (0,0.75) rectangle (1,2);
\fill[isolationoxide] (8.5,0.75) rectangle (11.5,2);
\fill[isolationoxide] (19,0.75) rectangle (20,2);
	\end{tikzpicture}
	\begin{tikzpicture}[node distance = 3cm, auto, thick,scale=\CrossAndTopSectionBig, every node/.style={transform shape}]
		% substrate
\fill[YellowOrange] (0,0) rectangle (20,12);
% trench area
\fill[DarkGray] (0,0) rectangle (1,12);
\fill[DarkGray] (8.5,0) rectangle (11.5,12);
\fill[DarkGray] (19,0) rectangle (20,12);
\fill[DarkGray] (0,0) rectangle (20,1.25);
\fill[DarkGray] (0,7.5) rectangle (20,12);
	\end{tikzpicture}
	\caption{Shallow trench isolation target geometry}
	\label{sti_target}
\end{figure}

As can be seen in \autoref{nwell_target}, the n-well and the STI trench are supposed to have approximately the same depth but the n-well and p-well go down a little bit further.
Because the n-well will be $\approx 4 \mu m$ in depth we have to match this with our trench depth.
I order to allow a sufficiently low resistance of the ESD diode but at the same time a sufficient isolation of between the standard cells a trade-ff has been done.
The targeted depth of the box isolation is $\approx 2 \mu m$.

\begin{figure}[H]
	\centering
	\begin{tikzpicture}[node distance =1cm, auto, thick,scale=\VLSILayout, every node/.style={transform shape}]
		\fill[nwell,opacity=0.2] (0.75,0.5) rectangle (8.75,7.75);
\fill[nwell,opacity=0.2] (11.25,0.5) rectangle (19.25,7.75);

\draw[dotted] (20.5,0.5) rectangle (25,6.5);

\node at (22.25,6) {\textbf{Layers}};

\fill[nwell,opacity=0.2] (21,1) rectangle (21.5,1.5);
\node at (22.25,1.25) {active};

\fill[resist,opacity=0.2] (21,1.5) rectangle (21.5,2);
\node at (22.25,1.75) {nwell};

\fill[blue,opacity=0.2] (21,2) rectangle (21.5,2.5);
\node at (22.25,2.25) {nimplant};

\fill[nitride,opacity=0.2] (21,2.5) rectangle (21.5,3);
\node at (22.25,2.75) {pimplant};

\fill[Emerald,opacity=0.2] (21,3) rectangle (21.5,3.5);
\node at (22.25,3.25) {gate};

\fill[Fuchsia,opacity=0.2] (21,3.5) rectangle (21.5,4);
\node at (22.25,3.75) {metal1};

\fill[DarkOrchid,opacity=0.2] (21,4) rectangle (21.5,4.5);
\node at (22.25,4.25) {via1};

\fill[LimeGreen,opacity=0.2] (21,4.5) rectangle (21.5,5);
\node at (22.25,4.75) {metal2};

\fill[ForestGreen,opacity=0.2] (21,5) rectangle (21.5,5.5);
\node at (22.25,5.25) {via2};

	\end{tikzpicture}
	\caption{Shallow trench isolation layout}
	\label{sti_layout}
\end{figure}

In \autoref{sti_layout} we can see the layout for the STI area.
The STI area will be everywhere, where no active areas are.
The field oxide needs to be grown out of trenches which can't be etched out of the silicon by using resist as a mask.
For that reason we will have to resort to a protective mask made from a silicon dioxide layer which has to be etched before hand.
So the mask will be exposed onto positive resist on top of the hard mask oxide layer in order to form a protective mask covering the active areas from having etched trenches into them.
After that we can either use a dry etching method or wet etching for cutting into the silicon substrate and making the active area become islands with trenches in between.
After these steps we have to remove the hard mask.
Our minimum width and height as well as the space between the active areas comes from the line space constrain of the silicon etcher and of course the optical limitations of the stepper which are as well 0.5\um.

\newpage

\subsection{Patterning}

The resist is being deposited using spin coating and then soft baked depending on the baking time for the specific resist.
The requirement is a \textbf{positive} tone resist.

\begin{figure}[H]
	\centering
	\begin{tikzpicture}[node distance = 3cm, auto, thick,scale=\CrossSectionOnly, every node/.style={transform shape}]
		% substrate
\fill[substrate] (0,0) rectangle (20,2);
\node at (2,0.5) {Silicon substrate};
%trenches
\fill[isolationoxide] (0,0.75) rectangle (1,2);
\fill[isolationoxide] (8.5,0.75) rectangle (11.5,2);
\fill[isolationoxide] (19,0.75) rectangle (20,2);
% n-well
\fill[nwell] (1.25,0.75) rectangle (8.5,2);
\node at (5.75,1) {N-Well};



	\end{tikzpicture}
	\drawStepArrow{Mask: sti}
	\begin{tikzpicture}[node distance = 3cm, auto, thick,scale=\CrossSectionOnly, every node/.style={transform shape}]
		% substrate
\fill[substrate] (0,0) rectangle (20,2);
\node at (2,0.5) {Silicon substrate};
%trenches
\fill[isolationoxide] (0,0.75) rectangle (1,2);
\fill[isolationoxide] (8.5,0.75) rectangle (11.5,2);
\fill[isolationoxide] (19,0.75) rectangle (20,2);
% n-well
\fill[nwell] (1.25,0.75) rectangle (8.5,2);
\node at (5.75,1) {N-Well};



% substrate islands
\fill[resist] (1.25,2.0) rectangle (8.25,4.0);
\fill[resist] (11.75,2.0) rectangle (18.75,4.0);

	\end{tikzpicture}
	\caption{Patterning with positive resist}
\end{figure}

The layout for being exposed onto the resist is being extracted from the "active" layer within the GDS2 file onto a  onto a \textbf{bright field} mask because we need to use the same mask again in \autoref{fox_chapter}, so alignment needs to be possible.

\subsection{Silicon etching}\label{sti_trench_etch}

Silicon can only be etched by a very aggressive chemical cocktail of  KOH and TMAH (20\%) or by plasma etching.

\begin{figure}[H]
	\centering
	\begin{tikzpicture}[node distance = 3cm, auto, thick,scale=\CrossSectionOnly, every node/.style={transform shape}]
		% substrate
\fill[substrate] (0,0) rectangle (20,1.9);
\node at (2,0.5) {Silicon substrate};

% substrate islands
\fill[substrate] (1,1.9) rectangle (8,2);
\fill[substrate] (11.5,1.9) rectangle (19,2);

% pad oxide
\fill[isolationoxide] (1,2) rectangle (8,2.6);
\fill[isolationoxide] (11.5,2) rectangle (19,2.6);
	\end{tikzpicture}
	\drawStepArrow{}
	\begin{tikzpicture}[node distance = 3cm, auto, thick,scale=\CrossSectionOnly, every node/.style={transform shape}]
		% substrate
\fill[substrate] (0,0) rectangle (20,1);
\node at (2,0.5) {Silicon substrate};

% substrate islands
\fill[substrate] (1,1) rectangle (8,2);
\fill[substrate] (11.5,1) rectangle (19,2);

% pad oxide
\fill[isolationoxide] (1,2) rectangle (8,2.6);
\fill[isolationoxide] (11.5,2) rectangle (19,2.6);
	\end{tikzpicture}
	\caption{Trench etching}
\end{figure}

\textbf{Possible approaches}:
\begin{itemize}
\item \textbf{"DRIE Etcher \#1" from HKUST} \\
This machine a normal etching rate of up to $2\frac{\mu m}{min}$ for etching silicon. \\
This means we etch for 1 minute in order to reach the desired depth. \\
The selectivity to oxide is >80:1 which means the etch speed for the hard mask will be at most $\frac{1}{80}2\frac{\mu m}{min}=\frac{1}{80}2000\frac{nm}{min}=25\frac{nm}{min}$.
\end{itemize}

\subsection{Resist removal}

Now we need to remove the contaminants for further processing.

\begin{figure}[H]
	\centering
	\begin{tikzpicture}[node distance = 3cm, auto, thick,scale=\CrossSectionOnly, every node/.style={transform shape}]
		% substrate
\fill[substrate] (0,0) rectangle (20,1);
\node at (2,0.5) {Silicon substrate};

% substrate islands
\fill[substrate] (1,1) rectangle (8,2);
\fill[substrate] (11.5,1) rectangle (19,2);

% pad oxide
\fill[isolationoxide] (1,2) rectangle (8,2.6);
\fill[isolationoxide] (11.5,2) rectangle (19,2.6);

	\end{tikzpicture}
	\drawStepArrow{}
	\begin{tikzpicture}[node distance = 3cm, auto, thick,scale=\CrossSectionOnly, every node/.style={transform shape}]
		% substrate
\fill[substrate] (0,0) rectangle (20,2);
\node at (2,0.5) {Silicon substrate};
%trenches
\fill[isolationoxide] (0,0.75) rectangle (1,2);
\fill[isolationoxide] (8.5,0.75) rectangle (11.5,2);
\fill[isolationoxide] (19,0.75) rectangle (20,2);

	\end{tikzpicture}
	\caption{Resist removal}
\end{figure}

We strip the resist, rinse and perform sulfuric cleaning.

\newpage

