`\section{P-well}\label{pwell_chapter}
In order to build CMOS on the same substrate, a P-well is required for building the complementary N-channel transistor for a n-p-channel logic circuitry.
The cross section as well as the top view of the targeted geometry are shown in \autoref{nwell_target}
\begin{figure}[H]
	\centering
	\begin{tikzpicture}[node distance = 3cm, auto, thick,scale=\CrossAndTopSectionBig, every node/.style={transform shape}]
		% substrate
\fill[substrate] (0,0) rectangle (20,2);
\node at (2,0.5) {Silicon substrate};
%trenches
\fill[isolationoxide] (0,0.75) rectangle (1,2);
\fill[isolationoxide] (8.5,0.75) rectangle (11.5,2);
\fill[isolationoxide] (19,0.75) rectangle (20,2);
% n-well
\fill[nwell] (1.25,0.75) rectangle (8.5,2);
\node at (5.75,1) {N-Well};


% p-well
\fill[pwell] (11.75,0.75) rectangle (18.75,2);
\node at (14.25,1) {P-Well};
	\end{tikzpicture}
	\begin{tikzpicture}[node distance = 3cm, auto, thick,scale=\CrossAndTopSectionBig, every node/.style={transform shape}]
		% substrate
\fill[YellowOrange] (0,0) rectangle (20,12);
% trench area
\fill[DarkGray] (0,0) rectangle (1,12);
\fill[DarkGray] (8.5,0) rectangle (11.5,12);
\fill[DarkGray] (19,0) rectangle (20,12);
\fill[DarkGray] (0,0) rectangle (20,1.25);
\fill[DarkGray] (0,7.5) rectangle (20,12);
\fill[nwell] (1.25,1) rectangle (8.25,7.25);
\fill[pwell] (11.75,1) rectangle (18.75,7.25);
	\end{tikzpicture}
	\caption{P-well target geometry}
	\label{pwell_target}
\end{figure}
The P-well will serve us as an island of higher p-doped substrate within the slightly p-doped basis substrate.

The dopant dose will be $2.5\times10^{12}cm^{-2}$ as calculated in the documentation of the process design leading to these steps\footnote{\url{https://github.com/leviathanch/libresiliconprocess/raw/master/process_design/process_design.pdf}}.

\begin{figure}[H]
	\centering
	\begin{tikzpicture}[node distance =1cm, auto, thick,scale=\VLSILayout, every node/.style={transform shape}]
		\fill[nwell,opacity=0.2] (0.75,0.5) rectangle (8.75,7.75);
\fill[nwell,opacity=0.2] (11.25,0.5) rectangle (19.25,7.75);

\draw[dotted] (20.5,0.5) rectangle (25,6.5);

\node at (22.25,6) {\textbf{Layers}};

\fill[nwell,opacity=0.2] (21,1) rectangle (21.5,1.5);
\node at (22.25,1.25) {active};

\fill[resist,opacity=0.2] (21,1.5) rectangle (21.5,2);
\node at (22.25,1.75) {nwell};

\fill[blue,opacity=0.2] (21,2) rectangle (21.5,2.5);
\node at (22.25,2.25) {nimplant};

\fill[nitride,opacity=0.2] (21,2.5) rectangle (21.5,3);
\node at (22.25,2.75) {pimplant};

\fill[Emerald,opacity=0.2] (21,3) rectangle (21.5,3.5);
\node at (22.25,3.25) {gate};

\fill[Fuchsia,opacity=0.2] (21,3.5) rectangle (21.5,4);
\node at (22.25,3.75) {metal1};

\fill[DarkOrchid,opacity=0.2] (21,4) rectangle (21.5,4.5);
\node at (22.25,4.25) {via1};

\fill[LimeGreen,opacity=0.2] (21,4.5) rectangle (21.5,5);
\node at (22.25,4.75) {metal2};

\fill[ForestGreen,opacity=0.2] (21,5) rectangle (21.5,5.5);
\node at (22.25,5.25) {via2};

\fill[nwell,opacity=\OpacityLayout] (1,0.75) rectangle (8.5,7.5);
\fill[pwell,opacity=\OpacityLayout] (11.5,0.75) rectangle (19,7.5);
	\end{tikzpicture}
	\caption{P-Well layout}
	\label{pwell_layout}
\end{figure}

In \autoref{pwell_layout} the layout of the P-well region on top of the active area region can be seen.

The p-well is being fit into the active area.

It should even be a little bit bigger than the active area, because of possible alignment offsets.

The layout is being automatically generated for GDS2 based on cifoutput rules, so you just have to draw you well.

\newpage

\subsection{Patterning}

The resist is being deposited spray or spin coating (spray coating is better because of the uneven surface!) and then soft baked depending on the baking time for the specific resist.
The layout for being exposed onto the resist is being extracted from the "pwell" layer within the GDS2 file onto a \textbf{bright field} mask.
The requirement is a \textbf{negative} tone resist.

\begin{figure}[H]
	\centering
	\begin{tikzpicture}[node distance = 3cm, auto, thick,scale=\CrossAndTopSection, every node/.style={transform shape}]
		% oxide
\fill[isolationoxide] (0,1.25) rectangle (20,2.75);

% oxide hill 1
\fill[isolationoxide] (1.25,2.75) rectangle (8.25,3.5);
\filldraw[line width=0, isolationoxide] (0.5,2.75) -- (1.25,2.75) -- (1.25,3.5);
\filldraw[line width=0, isolationoxide] (8.25,2.75) -- (8.25,3.5) -- (9.0,2.75);

% oxide hill 2
\fill[isolationoxide] (11.75,2.75) rectangle (18.75,3.5);
\filldraw[line width=0, isolationoxide] (11.0,2.75) -- (11.75,2.75) -- (11.75,3.5);
\filldraw[line width=0, isolationoxide] (18.75,2.75) -- (18.75,3.5) -- (19.5,2.75);

\node at (2,2.1) {SiO2};

\input{tikz_process_steps/nwell.a.tex}
	\end{tikzpicture}
	\begin{tikzpicture}[node distance = 3cm, auto, thick,scale=\CrossAndTopSection, every node/.style={transform shape}]
		% resist
\fill[isolationoxide] (0,0) rectangle (20,12);
	\end{tikzpicture}
	\drawStepArrow{Mask: pwell}
	\begin{tikzpicture}[node distance = 3cm, auto, thick,scale=\CrossAndTopSection, every node/.style={transform shape}]
		% resist
\fill[resist] (0.25,2.0) rectangle (11.5,5.0);
\fill[resist] (19,2.0) rectangle (19.75,5.0);

% oxide
\fill[isolationoxide] (0,1.25) rectangle (20,2.75);

% oxide hill 1
\fill[isolationoxide] (1.25,2.75) rectangle (8.25,3.5);
\filldraw[line width=0, isolationoxide] (0.5,2.75) -- (1.25,2.75) -- (1.25,3.5);
\filldraw[line width=0, isolationoxide] (8.25,2.75) -- (8.25,3.5) -- (9.0,2.75);

% oxide hill 2
\fill[isolationoxide] (11.75,2.75) rectangle (18.75,3.5);
\filldraw[line width=0, isolationoxide] (11.0,2.75) -- (11.75,2.75) -- (11.75,3.5);
\filldraw[line width=0, isolationoxide] (18.75,2.75) -- (18.75,3.5) -- (19.5,2.75);

\node at (2,2.1) {SiO2};

\input{tikz_process_steps/pwell.mask_dioxide_layer.a.tex}

	\end{tikzpicture}
	\begin{tikzpicture}[node distance = 3cm, auto, thick,scale=\CrossAndTopSection, every node/.style={transform shape}]
		% resist
\fill[resist] (0,0) rectangle (20,12);
% substrate
\fill[isolationoxide] (11.5,1.5) rectangle (19,7.25);
	\end{tikzpicture}
	\caption{Cross/top view of P-well layout on resist}
\end{figure}
The thickness of the resist layer and the baking duration will variate depending on the specific equipment for which this process will be implemented with.
Also after the exposure and development, the hard baking shouldn't be forgotten!

\subsection{Implantation}\label{pwell_implant_step}
We now need to inject the carriers into the upper level of the n-channel area so that we can later on drive them into the crystal during the drive-in step.

\begin{figure}[H]
	\centering
	\begin{tikzpicture}[node distance = 3cm, auto, thick,scale=\CrossSectionOnly, every node/.style={transform shape}]
		% oxide
\fill[isolationoxide] (0,1.25) rectangle (11.5,2.75);
\fill[isolationoxide] (19,1.25) rectangle (20,2.75);

% oxide hill 1
\fill[isolationoxide] (1.25,2.75) rectangle (8.25,3.5);
\filldraw[line width=0, isolationoxide] (0.5,2.75) -- (1.25,2.75) -- (1.25,3.5);
\filldraw[line width=0, isolationoxide] (8.25,2.75) -- (8.25,3.5) -- (9.0,2.75);

% oxide hill 2
\filldraw[line width=0, isolationoxide] (11.25,2.75) -- (11.5,2.75) -- (11.5,3.0);
\filldraw[line width=0, isolationoxide] (19.0,3.0)  -- (19.0,2.75) -- (19.25,2.75);

\node at (2,2.1) {SiO2};

\input{tikz_process_steps/sti.a.tex}
% n-well
\fill[nwell] (1.25,0.75) rectangle (8.5,2);
\node at (5.75,1) {N-Well};



\forloop{ct}{0}{\value{ct} < 21}
{
	\draw [->] (\value{ct},5) -- (\value{ct},4);
	\node at (\value{ct},5.2) {P$^{31}$};
}
	\end{tikzpicture}
	\drawStepArrow{Boron implant}
	\begin{tikzpicture}[node distance = 3cm, auto, thick,scale=\CrossSectionOnly, every node/.style={transform shape}]
		% resist
\fill[resist] (0.25,2.0) rectangle (11.5,5.0);
\fill[resist] (19,2.0) rectangle (19.75,5.0);

\input{tikz_process_steps/pwell.mask_dioxide_layer.b.tex}


% boron
\shade[upper left = pwell, upper right = pwell, lower right = substrate, lower left = substrate,] (11.5,1.5) rectangle (19.0,2);

	\end{tikzpicture} \\
	\caption{Doping process}
\end{figure}

\textbf{Possible approaches}:
\begin{itemize}
	\item \textbf{"CF-3000 Implanter (IMP-3000)" from HKUST} \\
	At HKUST we have an implanter which gives us better control over the initial surface concentration. \\
	These steps are needed to arrive with the desired geometry:
	The P-well is implanted with a Boron ($B^{11}$) dose of $2.5\times10^{12}cm^{-2}$ at an energy of 100 keV
\end{itemize}

\newpage

\subsection{Resist strip}

In order to avoid contamination of the machines we need to make sure all the resist has been stripped off from the wafer.

\begin{figure}[H]
	\centering
	\begin{tikzpicture}[node distance = 3cm, auto, thick,scale=\CrossSectionOnly, every node/.style={transform shape}]
		% resist
\fill[resist] (0.25,2.0) rectangle (11.5,5.0);
\fill[resist] (19,2.0) rectangle (19.75,5.0);

\input{tikz_process_steps/pwell.patterning.a.tex}


% boron
\shade[upper left = pwell, upper right = pwell, lower right = substrate, lower left = substrate,] (11.5,1.5) rectangle (19.0,2);


	\end{tikzpicture}
	\drawStepArrow{}
	\begin{tikzpicture}[node distance = 3cm, auto, thick,scale=\CrossSectionOnly, every node/.style={transform shape}]
		% substrate
\fill[substrate] (0,0) rectangle (20,1.25);
\node at (2,0.5) {Silicon substrate};
\fill[substrate] (0.25,1.25) rectangle (19.75,2);

% boron
\shade[upper left = pwell, upper right = pwell, lower right = substrate, lower left = substrate,] (11.5,1.5) rectangle (19.0,2);

	\end{tikzpicture}
	\caption{Resist removal}
\end{figure}
Please just use the solvent for the specific resist.

\newpage

