\section{n+ Implant}\label{nimplant}
For the bulk of the PMOS transistors and for the source and drain of the NMOS transistors highly doped  n+ areas are required.
In this step we're going to build these.

\begin{figure}[H]
	\centering
	\begin{tikzpicture}[node distance = 3cm, auto, thick,scale=\CrossAndTopSectionBig, every node/.style={transform shape}]
		% substrate
\fill[substrate] (0,0) rectangle (20,2);
\node at (2,0.5) {Silicon substrate};
%trenches
\fill[isolationoxide] (0,0.75) rectangle (1,2);
\fill[isolationoxide] (8.5,0.75) rectangle (11.5,2);
\fill[isolationoxide] (19,0.75) rectangle (20,2);
% n-well
\fill[nwell] (1,0.75) rectangle (8.5,2);
\node at (5.75,1) {N-Well};
\fill[nimplant] (1.5,1) rectangle (3,2);
\node at (2,1.5) {n+};
\fill[nimplant] (15,1) rectangle (16.5,2);
\node at (16,1.5) {n+};
\fill[nimplant] (12,1) rectangle (13.5,2);
\node at (13,1.5) {n+};
	\end{tikzpicture}
	\begin{tikzpicture}[node distance = 3cm, auto, thick,scale=\CrossAndTopSectionBig, every node/.style={transform shape}]
		\fill[substrate] (0,0) rectangle (20,10);

% n-well
\fill[nwell] (1,1.25) rectangle (8.5,7.5);

% p-well
\fill[pwell] (11.5,1.25) rectangle (19,7.5);

% n+
\fill[nimplant] (1.5,2) rectangle (3,6.5);
\fill[nimplant] (12,2) rectangle (13.5,6.5);
\fill[nimplant] (15,2) rectangle (16.5,6.5);

% trench area
\fill[isolationoxide] (0,0) rectangle (1,12);
\fill[isolationoxide] (8.5,0) rectangle (11.5,12);
\fill[isolationoxide] (19,0) rectangle (20,12);
\fill[isolationoxide] (0,0) rectangle (20,1.25);
\fill[isolationoxide] (0,7.5) rectangle (20,12);

% gate metal
\fill[gatemetal] (4.8,1.75) rectangle (6.7,9);
\fill[gatemetal] (13.3,1.75) rectangle (15.2,9);
\fill[gatemetal] (4.8,8) rectangle (15.2,9);
	\end{tikzpicture}
	\caption{N+ implant geometry target}
\end{figure}

The tricky thing here is to have a reasonable implant depth but not too deep because the deeper the junction, the higher the junction capacity which in turn limits the switching performance of the CMOS circuitry.

\begin{figure}[H]
	\centering
	\begin{tikzpicture}[node distance =1cm, auto, thick,scale=\VLSILayout, every node/.style={transform shape}]
		\input{tikz_process_steps/nimplant.layout.tex}

% p+
\fill[pimplant,opacity=\OpacityLayout] (3,0.75) rectangle (8.5,7.5);
\fill[pimplant,opacity=\OpacityLayout] (17,0.75) rectangle (19,7.5);
% gate metal
\fill[gatemetal,opacity=\OpacityLayout] (4.8,1.75) rectangle (6.7,8);
\fill[gatemetal,opacity=\OpacityLayout] (13.3,1.75) rectangle (15.2,8);
\fill[gatemetal,opacity=\OpacityLayout] (4.8,8) rectangle (15.2,10);


% n+
\fill[nimplant,opacity=\OpacityLayout] (1.5,2) rectangle (3,6.5);
\fill[nimplant,opacity=\OpacityLayout] (12,2) rectangle (16.5,6.5);

	\end{tikzpicture}
	\caption{N+ layout}
	\label{nimplant_layout}
\end{figure}

An example layout of p-implants can be seen in \autoref{nimplant_layout}, the mask is being extracted from the layer "n\_plus\_select".

Also important to notice is that this example layout is just for demonstration purposes only, please have a look at the standard cell documentation for the actual layouts. 

\newpage

\subsection{Mask dioxide layer}

In order to selectively inject charge carrying atoms into the crystalline structure a protective dioxide ($SiO_2$) layer needs to be grown on top of a p-type substrate.

\begin{figure}[H]
	\centering
	\begin{tikzpicture}[node distance = 3cm, auto, thick,scale=\CrossSectionOnly, every node/.style={transform shape}]
		\input{tikz_process_steps/nimplant.a.tex}
\fill[pimplant] (3.0,1.5) rectangle (5,2);
\node at (4,1.65) {p+};
\fill[pimplant] (6.5,1.5) rectangle (8.5,2);
\node at (7,1.65) {p+};
\fill[pimplant] (17,1.5) rectangle (18.75,2);
\node at (18,1.65) {p+};
\fill[gateoxide] (4.8,2) rectangle (6.7,2.3);
\fill[gateoxide] (13.3,2) rectangle (15.2,2.3);
\fill[gatemetal] (4.8,2.3) rectangle (6.7,2.6);
\fill[gatemetal] (13.3,2.3) rectangle (15.2,2.6);
	\end{tikzpicture}
	\drawStepArrow{}
	\begin{tikzpicture}[node distance = 3cm, auto, thick,scale=\CrossSectionOnly, every node/.style={transform shape}]
		% oxide
\fill[isolationoxide] (0,2.0) rectangle (20,4.0);

\fill[isolationoxide] (0,4.0) rectangle (0.5,4.75);
\fill[isolationoxide] (9.00,4.0) rectangle (11.0,4.75);
\fill[isolationoxide] (19.5,4.0) rectangle (20,4.75);

\filldraw[line width=0, isolationoxide] (0.5,4.75) -- (0.5,4.0) -- (1.25,4.0);
\filldraw[line width=0, isolationoxide] (8.5,4.0) -- (9.00,4.0) -- (9.00,4.75);

\filldraw[line width=0, isolationoxide] (11.0,4.75) -- (11.0,4.0) -- (11.75,4.0);
\filldraw[line width=0, isolationoxide] (18.75,4.0) -- (19.5,4.0) -- (19.5,4.75);

\input{tikz_process_steps/nimplant.a.tex}
\fill[pimplant] (3.0,1.5) rectangle (5,2);
\node at (4,1.65) {p+};
\fill[pimplant] (6.5,1.5) rectangle (8.5,2);
\node at (7,1.65) {p+};
\fill[pimplant] (17,1.5) rectangle (18.75,2);
\node at (18,1.65) {p+};
\fill[gateoxide] (4.8,2) rectangle (6.7,2.3);
\fill[gateoxide] (13.3,2) rectangle (15.2,2.3);
\fill[gatemetal] (4.8,2.3) rectangle (6.7,2.6);
\fill[gatemetal] (13.3,2.3) rectangle (15.2,2.6);
	\end{tikzpicture}
	\caption{Oxide layer}
\end{figure}

With an energy of 35keV for the implantation performed in \autoref{nimplant_implant_step}, the projected range of the dopants within the oxide will be 112nm (150nm tops) \footnote{\url{http://cleanroom.byu.edu/rangestraggle}}.
This means being on the safe side and having 200nm as the thickness is a good approach.
In order to grow the 200nm thick oxide layer, the wafer is being oxidized for around 14 minutes at 1050\degree C using wet oxidation which results in a dioxide layer of around 200nm in thickness\footnote{\url{http://cleanroom.byu.edu/OxideTimeCalc}}.

\subsection{Pattering}
\begin{figure}[H]
	\centering
	\begin{tikzpicture}[node distance = 3cm, auto, thick,scale=\CrossAndTopSection, every node/.style={transform shape}]
		% oxide
\fill[isolationoxide] (0,2.0) rectangle (20,4.0);

\fill[isolationoxide] (0,4.0) rectangle (0.5,4.75);
\fill[isolationoxide] (9.00,4.0) rectangle (11.0,4.75);
\fill[isolationoxide] (19.5,4.0) rectangle (20,4.75);

\filldraw[line width=0, isolationoxide] (0.5,4.75) -- (0.5,4.0) -- (1.25,4.0);
\filldraw[line width=0, isolationoxide] (8.5,4.0) -- (9.00,4.0) -- (9.00,4.75);

\filldraw[line width=0, isolationoxide] (11.0,4.75) -- (11.0,4.0) -- (11.75,4.0);
\filldraw[line width=0, isolationoxide] (18.75,4.0) -- (19.5,4.0) -- (19.5,4.75);

\input{tikz_process_steps/pimplant.a.tex}
\fill[gateoxide] (4.8,2) rectangle (6.7,2.3);
\fill[gateoxide] (13.3,2) rectangle (15.2,2.3);
\fill[gatemetal] (4.8,2.3) rectangle (6.7,2.6);
\fill[gatemetal] (13.3,2.3) rectangle (15.2,2.6);
	\end{tikzpicture}
	\begin{tikzpicture}[node distance = 3cm, auto, thick,scale=\CrossAndTopSection, every node/.style={transform shape}]
		\fill[isolationoxide] (0,0) rectangle (20,12);
	\end{tikzpicture}
	\drawStepArrow{}
	\begin{tikzpicture}[node distance = 3cm, auto, thick,scale=\CrossAndTopSection, every node/.style={transform shape}]
		% resist
\fill[resist] (0,2.0) rectangle (0.75,5.0);
\fill[resist] (3.25,2.0) rectangle (11.25,5.0);
\fill[resist] (16.25,2.0) rectangle (20,5.0);

% oxide
\fill[isolationoxide] (0,2.0) rectangle (20,4.0);

\fill[isolationoxide] (0,4.0) rectangle (0.5,4.75);
\fill[isolationoxide] (9.00,4.0) rectangle (11.0,4.75);
\fill[isolationoxide] (19.5,4.0) rectangle (20,4.75);

\filldraw[line width=0, isolationoxide] (0.5,4.75) -- (0.5,4.0) -- (1.25,4.0);
\filldraw[line width=0, isolationoxide] (8.5,4.0) -- (9.00,4.0) -- (9.00,4.75);

\filldraw[line width=0, isolationoxide] (11.0,4.75) -- (11.0,4.0) -- (11.75,4.0);
\filldraw[line width=0, isolationoxide] (18.75,4.0) -- (19.5,4.0) -- (19.5,4.75);

\input{tikz_process_steps/gate.a.tex}

	\end{tikzpicture}
	\begin{tikzpicture}[node distance = 3cm, auto, thick,scale=\CrossAndTopSection, every node/.style={transform shape}]
		\fill[resist] (0,0) rectangle (20,12);

% n+
\fill[isolationoxide] (1.5,2) rectangle (3,6.5);
\fill[isolationoxide] (12,2) rectangle (13.5,6.5);
\fill[isolationoxide] (15,2) rectangle (16.5,6.5);
	\end{tikzpicture}
	\caption{N+ region resist mask}
\end{figure}

\subsection{Etching}
\begin{figure}[H]
	\centering
	\begin{tikzpicture}[node distance = 3cm, auto, thick,scale=\CrossAndTopSection, every node/.style={transform shape}]
		\input{tikz_process_steps/sti.a.tex}
% n-well
\fill[nwell] (1,0.75) rectangle (8.5,2);
\node at (5.75,1) {N-Well};
% gate oxide
\fill[LightGray] (4.8,2) rectangle (6.7,2.3);
\fill[LightGray] (13.3,2) rectangle (15.2,2.3);
% gate poly
\fill[BrickRed] (4.8,2.3) rectangle (6.7,2.6);
\fill[BrickRed] (13.3,2.3) rectangle (15.2,2.6);
% oxide
\fill[gray] (0,2) rectangle (4.8,2.6);
\fill[gray] (6.7,2) rectangle (13.3,2.6);
\fill[gray] (15.2,2) rectangle (20,2.6);
\fill[gray] (0,2.6) rectangle (20,3);
% resist
\fill[orange] (0,3) rectangle (1.5,3.6);
\fill[orange] (3,3) rectangle (12,3.6);
\fill[orange] (16.5,3) rectangle (20,3.6);;
	\end{tikzpicture}
	\begin{tikzpicture}[node distance = 3cm, auto, thick,scale=\CrossAndTopSection, every node/.style={transform shape}]
		\fill[orange] (0,0) rectangle (20,12);

% n+
\fill[gray] (1.5,2) rectangle (3,6.5);
\fill[gray] (12,2) rectangle (13.5,6.5);
\fill[gray] (15,2) rectangle (16.5,6.5);
	\end{tikzpicture}
	\drawStepArrow{}
	\begin{tikzpicture}[node distance = 3cm, auto, thick,scale=\CrossAndTopSection, every node/.style={transform shape}]
		% oxide
\fill[isolationoxide] (0,2) rectangle (1.25,3.5);
\fill[isolationoxide] (3,2) rectangle (11.75,3.5);
\fill[isolationoxide] (17,2) rectangle (20,3.5);

% resist
\fill[resist] (0,3.5) rectangle (1.25,4.1);
\fill[resist] (3,3.5) rectangle (11.75,4.1);
\fill[resist] (17,3.5) rectangle (20,4.1);

\input{tikz_process_steps/nimplant.a.tex}
\fill[pimplant] (3.0,1.5) rectangle (5,2);
\node at (4,1.65) {p+};
\fill[pimplant] (6.5,1.5) rectangle (8.5,2);
\node at (7,1.65) {p+};
\fill[pimplant] (17,1.5) rectangle (18.75,2);
\node at (18,1.65) {p+};
\fill[gateoxide] (4.8,2) rectangle (6.7,2.3);
\fill[gateoxide] (13.3,2) rectangle (15.2,2.3);
\fill[gatemetal] (4.8,2.3) rectangle (6.7,2.6);
\fill[gatemetal] (13.3,2.3) rectangle (15.2,2.6);
	\end{tikzpicture}
	\begin{tikzpicture}[node distance = 3cm, auto, thick,scale=\CrossAndTopSection, every node/.style={transform shape}]
		\fill[orange] (0,0) rectangle (20,12);

% n+
\fill[Goldenrod] (1.5,2) rectangle (3,6.5);
\fill[YellowOrange] (12,2) rectangle (13.5,6.5);
\fill[YellowOrange] (15,2) rectangle (16.5,6.5);
	\end{tikzpicture}
	\caption{N+ region opened}
\end{figure}

\subsection{Cleaning}
\begin{figure}[H]
	\centering
	\begin{tikzpicture}[node distance = 3cm, auto, thick,scale=\CrossSectionOnly, every node/.style={transform shape}]
		% oxide
\fill[isolationoxide] (0,2) rectangle (1.25,3.5);
\fill[isolationoxide] (3,2) rectangle (11.75,3.5);
\fill[isolationoxide] (17,2) rectangle (20,3.5);

% resist
\fill[resist] (0,3.5) rectangle (1.25,4.1);
\fill[resist] (3,3.5) rectangle (11.75,4.1);
\fill[resist] (17,3.5) rectangle (20,4.1);

\input{tikz_process_steps/nimplant.a.tex}
\fill[pimplant] (3.0,1.5) rectangle (5,2);
\node at (4,1.65) {p+};
\fill[pimplant] (6.5,1.5) rectangle (8.5,2);
\node at (7,1.65) {p+};
\fill[pimplant] (17,1.5) rectangle (18.75,2);
\node at (18,1.65) {p+};
\fill[gateoxide] (4.8,2) rectangle (6.7,2.3);
\fill[gateoxide] (13.3,2) rectangle (15.2,2.3);
\fill[gatemetal] (4.8,2.3) rectangle (6.7,2.6);
\fill[gatemetal] (13.3,2.3) rectangle (15.2,2.6);
	\end{tikzpicture}
	\drawStepArrow{}
	\begin{tikzpicture}[node distance = 3cm, auto, thick,scale=\CrossSectionOnly, every node/.style={transform shape}]
		% substrate
\fill[substrate] (0,0) rectangle (20,2);
\node at (2,0.5) {Silicon substrate};
%trenches
\fill[isolationoxide] (0,0.75) rectangle (1,2);
\fill[isolationoxide] (8.5,0.75) rectangle (11.5,2);
\fill[isolationoxide] (19,0.75) rectangle (20,2);
% n-well
\fill[nwell] (1,0.75) rectangle (8.5,2);
\node at (5.75,1) {N-Well};
% oxide
\fill[gray] (0,2) rectangle (3.5,3);
\fill[gray] (5,2) rectangle (6.5,3);
\fill[gray] (8,2) rectangle (17,3);
\fill[gray] (18.5,2) rectangle (20,3);
	\end{tikzpicture}
	\caption{Resist removal}
\end{figure}

\subsection{Implantation/Doping}\label{nimplant_implant_step}

We now need to bring in the carriers in order to build the n-junctions.

\begin{figure}[H]
	\centering
	\begin{tikzpicture}[node distance = 3cm, auto, thick,scale=\CrossSectionOnly, every node/.style={transform shape}]
		% oxide
\fill[isolationoxide] (0,2) rectangle (1.5,3.5);
\fill[isolationoxide] (3,2) rectangle (12,3.5);
\fill[isolationoxide] (16.5,2) rectangle (20,3.5);

\forloop{ct}{0}{\value{ct} < 21}
{
	\draw [->] (\value{ct},5) -- (\value{ct},4);
	\node at (\value{ct},5.2) {P$^{31}$};
}

\input{tikz_process_steps/nimplant.a.tex}
\fill[pimplant] (3.0,1.5) rectangle (5,2);
\node at (4,1.65) {p+};
\fill[pimplant] (6.5,1.5) rectangle (8.5,2);
\node at (7,1.65) {p+};
\fill[pimplant] (17,1.5) rectangle (18.75,2);
\node at (18,1.65) {p+};
\fill[gateoxide] (4.8,2) rectangle (6.7,2.3);
\fill[gateoxide] (13.3,2) rectangle (15.2,2.3);
\fill[gatemetal] (4.8,2.3) rectangle (6.7,2.6);
\fill[gatemetal] (13.3,2.3) rectangle (15.2,2.6);
	\end{tikzpicture}
	\drawStepArrow{}
	\begin{tikzpicture}[node distance = 3cm, auto, thick,scale=\CrossSectionOnly, every node/.style={transform shape}]
		% oxide
\fill[isolationoxide] (0,2) rectangle (1.5,3.5);
\fill[isolationoxide] (3,2) rectangle (12,3.5);
\fill[isolationoxide] (16.5,2) rectangle (20,3.5);

\input{tikz_process_steps/nimplant.a.tex}
\fill[pimplant] (3.0,1.5) rectangle (5,2);
\node at (4,1.65) {p+};
\fill[pimplant] (6.5,1.5) rectangle (8.5,2);
\node at (7,1.65) {p+};
\fill[pimplant] (17,1.5) rectangle (18.75,2);
\node at (18,1.65) {p+};
\fill[gateoxide] (4.8,2) rectangle (6.7,2.3);
\fill[gateoxide] (13.3,2) rectangle (15.2,2.3);
\fill[gatemetal] (4.8,2.3) rectangle (6.7,2.6);
\fill[gatemetal] (13.3,2.3) rectangle (15.2,2.6);

\fill[nimplant] (1.5,1) rectangle (3,2);
\node at (2,1.5) {n+};
\fill[nimplant] (15,1) rectangle (16.5,2);
\node at (16,1.5) {n+};
\fill[nimplant] (12,1) rectangle (13.5,2);
\node at (13,1.5) {n+};
	\end{tikzpicture}
	\caption{N+ injection process}
\end{figure}

\textbf{Possible approaches}:
\begin{itemize}
	\item \textbf{"CF-3000 Implanter (IMP-3000)" from HKUST} \\
	At HKUST we have an implanter which gives us better control over the initial surface concentration. \\
	These steps are needed to arrive with the desired geometry:
	\begin{enumerate}
		\item Preparing by default cleaning
		\item The N-well is implanted with a Phosphorus ($P^{31}$) dose of $2.5\times10^{12}cm^{-2}$ at an energy of 100 keV.
		\item The N-well is annealed for 30 minutes at 1050\degreesC in $N_2$ environment (DIF-A1)\\
		After that the P-well will be around 2\um deep and the N-Well around 1\um deep
	\end{enumerate}
	\item \textbf{Constant source diffusion} \\
	We can add a layer of Phosphorus solution and diffusing in order to have an initial concentration in order to reach the desired concentration later by main diffusion.
\end{itemize}

\subsection{Oxide removal}
\begin{figure}[H]
	\centering
	\begin{tikzpicture}[node distance = 3cm, auto, thick,scale=\CrossSectionOnly, every node/.style={transform shape}]
		% oxide
\fill[isolationoxide] (0,2) rectangle (1.5,3.5);
\fill[isolationoxide] (3,2) rectangle (12,3.5);
\fill[isolationoxide] (16.5,2) rectangle (20,3.5);

\input{tikz_process_steps/nimplant.a.tex}
\fill[pimplant] (3.0,1.5) rectangle (5,2);
\node at (4,1.65) {p+};
\fill[pimplant] (6.5,1.5) rectangle (8.5,2);
\node at (7,1.65) {p+};
\fill[pimplant] (17,1.5) rectangle (18.75,2);
\node at (18,1.65) {p+};
\fill[gateoxide] (4.8,2) rectangle (6.7,2.3);
\fill[gateoxide] (13.3,2) rectangle (15.2,2.3);
\fill[gatemetal] (4.8,2.3) rectangle (6.7,2.6);
\fill[gatemetal] (13.3,2.3) rectangle (15.2,2.6);

\fill[nimplant] (1.5,1) rectangle (3,2);
\node at (2,1.5) {n+};
\fill[nimplant] (15,1) rectangle (16.5,2);
\node at (16,1.5) {n+};
\fill[nimplant] (12,1) rectangle (13.5,2);
\node at (13,1.5) {n+};
	\end{tikzpicture}
	\drawStepArrow{}
	\begin{tikzpicture}[node distance = 3cm, auto, thick,scale=\CrossSectionOnly, every node/.style={transform shape}]
		% substrate
\fill[substrate] (0,0) rectangle (20,2);
\node at (2,0.5) {Silicon substrate};
%trenches
\fill[isolationoxide] (0,0.75) rectangle (1,2);
\fill[isolationoxide] (8.5,0.75) rectangle (11.5,2);
\fill[isolationoxide] (19,0.75) rectangle (20,2);
% n-well
\fill[nwell] (1,0.75) rectangle (8.5,2);
\node at (5.75,1) {N-Well};
\fill[nimplant] (1.5,1) rectangle (3,2);
\node at (2,1.5) {n+};
\fill[nimplant] (15,1) rectangle (16.5,2);
\node at (16,1.5) {n+};
\fill[nimplant] (12,1) rectangle (13.5,2);
\node at (13,1.5) {n+};
	\end{tikzpicture}
	\caption{Oxide removal}
\end{figure}
