\tikzstyle{block} = [rectangle, draw, fill=blue!20, text width=3cm, text centered, rounded corners, minimum height=1.5cm]
\tikzstyle{line} = [draw, very thick, color=black!50, -latex']

The general flow chart of the overall process flow can be seen in \autoref{full_flow}.
These process steps will be discussed within the following sections.
\begin{figure}[H]
	\centering
	\begin{tikzpicture}[node distance=2cm, thick,scale=0.8, every node/.style={transform shape}]
		%% Place nodes
		%active CMOS				
		\node [block] (isolation) at (4,20) {Isolation (STI)\\ \autoref{sti}};
		\node [block, below of=isolation] (nwell) {N-Well\\ \autoref{nwell_chapter}};
		\node [block, below of=nwell] (pwell) {P-Well\\ \autoref{pwell_chapter}};
		\node [block, below of=pwell] (gate) {Gate\\ \autoref{gate}};
		\node [block, below of=gate] (np) {n+ Implant\\ \autoref{nimplant}};
		\node [block, below of=np] (pp) {p+ Implant\\ \autoref{pimplant}};
		\node [block, below of=pp] (silicification) {Silicification\\ \autoref{step_silicification}};
		%post proces
		\node [block] (via1) at (8,8) {First vias\\ \autoref{via1}};
		\node [block, above of=via1] (metal1) {First metal\\ \autoref{metal1}};
		\node [block, above of=metal1] (via2) {Additional vias\\ \autoref{via2}};
		\node [block, above of=via2] (metal2) {Additional metal\\ \autoref{metal2}};
		\node (repeat) at (10.5,14.5) {Repeat};

		%% Draw edges
		\path [line] (isolation) -- (nwell);
		\path [line] (nwell) -- (pwell);
		\path [line] (pwell) -- (gate);
		\path [line] (gate) -- (np);
		\path [line] (np) -- (pp);
		\path [line] (pp) -- (silicification);
		\path [line] (silicification) -- (via1);
		\path [line] (via1) -- (metal1);
		\path [line] (metal1) -- (via2);
		\path [line] (via2) -- (metal2);
		\path [line] (metal2) -- +(3,0) -- +(3,-2) -- (via2);

		\draw[dotted] (2,6) rectangle (6,21);
		\node at (4,6.5) {CMOS process};
		\draw[dotted] (6,6) rectangle (12,21);
		\node at (8,6.5) {Interconnect};

		%\draw[dotted] (1.5,9) rectangle (10.5,21.5);
		%\node at (4,9.5) {Front-end processing};

		%\draw[dotted] (11,9) rectangle (15,21.5);
		%\node at (13,9.5) {Back-end processing};
	\end{tikzpicture}
	\caption{Frontend and backend process flow}
	\label{full_flow}
\end{figure}
The six overall process steps are part of an active part of the technology, while the final metal (respectively contact) layers will be used for making a contact between the logic gates and macro cells and making them available to the exterior world.

For this process p-substrate is the required basic substrate, but forks and modifications will be very well possible based on a Graphene substrate or alike, still under the LSPL.
The starting material is a p-type, <100> oriented silicon with a doping concentration of $\approx 9\times10^{14}cm^{-3}$.\\



